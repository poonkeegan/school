\documentclass{article}
\usepackage[utf8]{inputenc}
\usepackage[margin=1in]{geometry}
\usepackage{amsmath, amsfonts}
\usepackage{fancyhdr}
\usepackage{multicol}
\usepackage{graphicx}
\usepackage{amsthm}
\usepackage{pgfplotstable}
\usepackage{longtable}
\usepackage{algorithm2e}
\graphicspath{ {images/} }
\pagestyle{empty}
\fancyhf{}
\cfoot{\thepage}

\newtheorem*{lemma}{Lemma}
\newtheorem*{theorem}{Theorem}
\newcommand{\twonorm}[1]{\| #1 \|_2}
\newcommand{\rdim}[2]{\mathbb{R}^{#1 \times #2}}
\newcommand{\R}{\mathbb{R}}
\newcommand{\deriv}[1]{\frac{d}{d #1}}
\newcommand{\parti}[1]{\frac{\partial}{\partial #1}}
\newcommand{\partis}[2]{\frac{\partial #2}{\partial #1}}
\lhead{CSCD37: Assignment \#2 }
\rhead{
Poon, Keegan\\
poonkeeg\\
1002423727}
\pagenumbering{gobble}
\renewcommand{\headrulewidth}{0pt}
\begin{document}

\thispagestyle{fancy}
\begin{enumerate}
    \item Consider the IVODE $y'' = y$ for $t \geq 0$, with initial values
        $y(0) = 1$ and $y'(0) = 2$.
        \begin{enumerate}
            \item Express this second-order ODE as an equivalent system of
                two first-order ODEs.

                The system of first-order ODEs is given by $y_1 = y$ and 
                $y_2 = y'$. Then the system is $y_1' = y_2$ and $y_2' =
                y_1$.

            \item What are the corresponding inital values for the system
                of ODEs in (a)?

                Since $y = y_1$ and $y' = y_2$ the initial values are just
                $y_1(0) = 1$ and $y_2(0) = 2$.

            \item Are solutions of this system stable? Justify your answer.

                We have that $y_1' = f_1(t,y_2)$ and $y_2' = f_2(t,y_1)$, 
                so analyzing the eigenvalues will give the stability 
                of the system. First compute the matrix.
                \begin{align*} 
                    \begin{bmatrix}
                        \partis{y_1}{f_1} & \partis{y_2}{f_1} \\
                        \partis{y_1}{f_2} & \partis{y_2}{f_2}
                    \end{bmatrix} 
                    &=
                    \begin{bmatrix}
                        0 & 1 \\
                        1 & 0
                    \end{bmatrix} 
                \end{align*}
                Now to evaluate the eigenvalues of the matrix.
                \begin{align*} 
                    \begin{bmatrix}
                        0 & 1 \\
                        1 & 0
                    \end{bmatrix} - \lambda I &= 
                    \begin{bmatrix}
                        -\lambda & 1 \\
                        1 & -\lambda
                    \end{bmatrix} \\
                    \implies 
                    \begin{vmatrix}
                        -\lambda & 1 \\
                        1 & -\lambda
                    \end{vmatrix}
                    &= (\lambda^2 - 1) = 0 \\
                    \implies \lambda &= \pm 1
                \end{align*}
                There is both a positive and negative eigenvalue so it is
                unstable in the general case, unless the interval of 
                integration is sufficiently small. In that case, the 
                instability won't start to effect the solution as much.
                 
            \item Perform one step of forward Euler method for this ODE
                system using a step size of $h = 0.5$.

            \begin{align*} 
                y_k &\approx y(0) \\
                y_{k+1} &= y_k + h_kf(t_k,y_k) \\
                \implies y_{k+1} &= 1 + (0.5)f(0,1) \\
                \implies y_{k+1} &= 1 + (0.5)y' \\
            \end{align*}
            \item Is the forward Euler method stable for this problem using
                this step size? Justify your answer.
                
                The forward Euler method is not stable here since the
                equation $\twonorm{I + hf_y}$ of the amplifcation matrix
                is $> 1$.

            \item Is the backward Euler method stable for this problem using
                this step size? Justify your answer.

                The backward Euler method is stable for the problem
        \end{enumerate}
    \item Consider the Trapezoidal Rule
        \begin{equation}
            y_{k+1} = y_k + \frac{h_k}{2}[f(t_k,y_k) + f(t_{k+1}, y_{k+1})],
            \, k = 0,1,2,\dots
        \end{equation}
        for integrating the general IVODE $y'(t) = f(t,y(t)),\,
        y(t_0) = y_0$.
        \begin{enumerate}
            \item Show how (1) is derived by combining two appropriate 
                Taylor expansions. What is the truncation error (local 
                error) in (1)?
                
                From FEM and BEM we have.
               \[ y(t + h) = y(t) + hy'(t) + \frac{h^2}{2!} y''(t)
               + \frac{h^3}{3!}y'''(\psi_1)\]
               \[ y([t + h] - h) = y(t + h) - hy'(t + h) 
               + \frac{(-h)^2}{2!} y''(t+h) 
               + \frac{(-h)^3}{3!}y'''(\psi_2) \]
               
               Adding them together gives 
               \begin{align*}
               2y(t + h) - hy'(t + h) 
               + \frac{(-h)^2}{2!} y''(t+h) 
               + \frac{(-h)^3}{3!}y'''(\psi_2)
               = 2y(t) + hy'(t) + \frac{h^2}{2!} y''(t) 
               + \frac{h^3}{3!}y'''(\psi_1) \\
               2y(t + h) 
               = 2y(t) + hy'(t) + hy'(t + h) + \frac{h^2}{2!} y''(t) 
               - \frac{h^2}{2!} y''(t+h) 
               + \frac{h^3}{3!}y'''(\psi_1) 
               + \frac{h^3}{3!}y'''(\psi_2) \\
               y(t + h) 
               = y(t) + \frac{h}{2}
               [y'(t) + y'(t + h) + \frac{h}{2!} y''(t) 
               - \frac{h}{2!} y''(t+h) 
               + \frac{h^2}{3!}y'''(\psi_1) 
               + \frac{h^2}{3!}y'''(\psi_2)] \\
               y(t + h) \approx y(t) + \frac{h}{2}
               [y'(t) + y'(t + h) 
               + \frac{h^2}{3!}y'''(\psi_1) 
               + \frac{h^2}{3!}y'''(\psi_2)] \text{ Since as }
               h \rightarrow 0,\, t + h \approx t\\
               \text{Where } \psi_1 \in [t,\, t+h] \& \,
               \psi_2 \in [t+h ,\, t] \\
               y(t + h) \approx y(t) + \frac{h}{2}
               [y'(t) + y'(t + h)]
               + \frac{h^3}{3!}y'''(\psi) 
               \text{Where } \psi \in [t,\, t+h] \& \,
               \end{align*}
               Cutting this equation off at the first two terms, gives
               the approxmiation
               \[ y_{k+1}
               = y_k + \frac{h}{2}
               [f(t_k, y_k) + f(t_{k+1}, y_{k+1})] \]
               and this has the evident truncation error of 
               \[\frac{h^3}{3!}y'''(\psi) \]
            \item Derive the global error propagation formula for (1),
                showing how the global error at $t_{k+1}$ is related
                to the global error at $t_k$ and the local error $t_k$.
                Recall that this is a \textit{general} IVP with $y,f:
                \mathbb{R}^n \rightarrow \mathbb{R}^n$.
                \begin{align*} y(t_{k+1}) - y_{k+1} = 
                    y'(t_k) + \frac{h}{2}
                    [f(t_k, y(t_k)) + f(t_{k+1}, y(t_{k+1})) ]
                    - y_k - \frac{h}{2}[f(t_k, k)
                    + f(t_{k+1}, y_{k+1})] + \frac{h^3}{3!}y'''(\psi) \\
                    \text{Using MVT to simplify} \\
                    y(t_{k+1}) - y_{k+1} = 
                    y(t_k) - y_k + \frac{h}{2}
                    [f_y (t_k, \eta_k)(y(t_k) - y_k) 
                    + f_y (t_{k+1}, \eta_{k+1})(y(t_{k+1}) - y_{k+1})]
                    + \frac{h^3}{3!}y'''(\psi) \\
                    [I + f_y (t_{k+1}, \eta_{k+1})](y(t_{k+1}) - y_{k+1}) 
                    = \frac{h}{2}
                    [I + f_y (t_k, \eta_k)](y(t_k) - y_k)
                    + \frac{h^3}{3!}y'''(\psi) \\
                    (y(t_{k+1}) - y_{k+1}) 
                    = \frac{h}{2}
                    [I + f_y (t_{k+1}, \eta_{k+1})]^{-1}
                    [I + f_y (t_k, \eta_k)](y(t_k) - y_k)
                    + [I + f_y (t_{k+1}, \eta_{k+1})]^{-1}
                    + \frac{h^3}{3!}y'''(\psi) \\
                \end{align*}
            \item When applied to the model problem $y'(t) = \lambda y(t),
                y: \mathbb{R} \rightarrow \mathbb{R}, \lambda \in 
                \mathbb{R},$ (1) simplifies. Write out the simplified form.
                \begin{align*}
                y_{k+1} &= y_k + \frac{h_k}{2}[f(t_k,y_k) 
                + f(t_{k+1}, y_{k+1})] \\
                y_{k+1} &= y_k + \frac{h_k}{2}[\lambda y_k
                + \lambda y_{k+1}] \\
                y_{k+1} - \lambda y_{k+1} &= 
                y_k \bigg( 1 +  \frac{h_k}{2}\lambda \bigg) \\
                y_{k+1} &= y_k \bigg( \frac{2 + h_k}{2(1-\lambda)}\bigg) \\
                y_{k+1} &= y_0 \bigg( \frac{2 + h_k}{2(1-\lambda)}
                \bigg)^k \\
                \end{align*}
            \item Derive the growth factor for the simplified form.
            \item Using the growth factor for the simplified form derived
                in (d), define and sketch the region of absolute stability
                for (1). \textit{Show all of your work.}
            \item Is (1) A-stable? Is it L-stable? \textit{Justify your 
                answer}.
            \item (1) clearly is more expensive per step than both the
                forward Euler method (FEM) and backward Euler method
                (BEM) discussed in lecture. What advantage(s), if any,
                does (1) have over FEM and BEM?
        \end{enumerate}
\end{enumerate}
\end{document}
