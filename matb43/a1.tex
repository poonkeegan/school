\documentclass{article}
\usepackage[utf8]{inputenc}
\usepackage[margin=1in]{geometry}
\usepackage{amsmath, amsfonts}
\usepackage{fancyhdr}
\usepackage{multicol}
\usepackage{graphicx}
\usepackage{amsthm}
\graphicspath{ {images/} }
\pagestyle{empty}
\fancyhf{}
\cfoot{\thepage}

\newtheorem*{lemma}{Lemma}

\lhead{MATB43: Assignment \#1 \\
        TUT 3}
\rhead{
Poon, Keegan\\
1002423727}

\renewcommand{\headrulewidth}{0pt}
\begin{document}

\thispagestyle{fancy}
\begin{enumerate}
    \item Let $S \subset \mathbb{R}$ be bounded above, and let $a =\text{sup }S$. Prove that for any $\epsilon>0$, there exists $x \in S$ such that $x > a - \epsilon$.
    \begin{proof} By contradiction: Suppose there is some $\epsilon > 0$ such that $\forall x \in S,\: x \leq a - \epsilon$.

    This means, by definition $S$ is bounded by $a - \epsilon$, but $a > a - \epsilon$, which contradicts that $a = \text{sup}\ S$. Therefore there must be an element in the range $(a - \epsilon, a)$ in $S$, $\forall \epsilon$.
    \end{proof}

\item Show that sup$\{r \in \mathbb{Q}|r<a\} = a$, for any $a \in \mathbb{R}$.
\begin{proof}
By definition, $S$ is bounded by $a$ from above ($\forall r \in S = \{r \in \mathbb{Q}|r<a\}, r < a$). Need to show it is the smallest such bound, i.e. $\forall k \in \mathbb{R}, k < a \implies \exists l \in S \text{ s.t. } k < l < a$. Let $k \in \mathbb{R}, k < a$. Then since $\mathbb{Q}$ dense in $\mathbb{R}$, there is some $q \in \mathbb{Q}$ such that $k < q < a$, and so $q \in S$. So $a$ is the lowest number that can serve as a bound, so $a = \text{sup } S$.
\end{proof}

\item  Let $a_1 = 1$ and let $a_{n+1} = (1 - 1/4n^2)a_n$, for any $n \geq 1$.
    \begin{enumerate}
        \item Show that lim $a_n$ exists.
        
        To show the limit exists, I will use the fact that a bounded monotone sequence converges.
        \begin{proof}
        Monotone (Strictly Decreasing): Base $a_1$ = 1, $a_2 = 3/4$, $1 > 3/4$.

        Induction $:$ Suppose $a_{n-1} > a_n$, $ n \in \mathbb{N}$
        \begin{align*}
        \Big(1 - \frac{1}{4(n-1)^2} \Big) a_{n-1} &> \Big(1 - \frac{1}{4(n-1)^2}\Big)a_{n} \\
        \Big(1 - \frac{1}{4(n-1)^2} \Big) a_{n-1} &> \Big(1 - \frac{1}{4(n-1)^2}\Big)a_{n} > \Big(1 - \frac{1}{4(n)^2}\Big)a_{n} \\
        \Big(1 - \frac{1}{4(n-1)^2} \Big) a_{n-1} &> \Big(1 - \frac{1}{4n^2}\Big)a_{n} \\
        a_{n} > a_{n+1}
        \end{align*}

        Bounded: WTP: $0 < a_{k} \leq 1, \: \forall k$ Base $a_0 = 1 \in (0,1]$

        Induction $:$ Suppose $0 < a_n \leq 1, n \in \mathbb{N} $

        \begin{align*}
            0 &< a_n \leq 1 \\
            0 &< \Big(1 - \frac{1}{4n^2} \Big)a_n \leq \Big(1 - \frac{1}{4n^2} \Big) \\
            0 &<a_{n+1} \leq \Big(1 - \frac{1}{4n^2} \Big) < 1 \\
            0 &<a_{n+1}  < 1
        \end{align*}
        
        Since the sequence is monotone and bounded, it converges.

        \end{proof}
        \item What do you think that lim $a_n$ is?

        $\approx 0.6$ from empirical testing.
    \end{enumerate}

    \item Let $\{s_n\}$ be a sequence such that
   \[|s_{n+1} - s_n| < 2^{-N}\] 
   for all $n \in \mathbb{N}$. Prove that $\{ s_n \}$ is a Cauchy sequence and hence converges.
    
    Want $\forall \epsilon ,\: \exists N$ s.t. $\forall m,n > N \implies |x_m - x_n| < \epsilon$
    
    \begin{lemma}
        Let $N \in \mathbb{N}$, $\forall m, n \in \mathbb{N}, m,n > N\implies |s_m - s_n| < 2^{-N + 1}$
        \begin{proof}
            Given $N$, choose $N < m<n$ such that the distance $|s_n - s_m|$ is the greatest possible for any choice of $m,n$. Then:
            \begin{align*}
                |s_n - s_m| &= |s_n + \sum_{i = m + 1}^{n-1}(s_i - s_i) - s_m| \\
                &= |s_n - s_{n-1} + \sum_{i=m+1}^{n-2}(s_{i+1} - s_i) + s_{m+1} - s_m| \\
                &= |\sum_{i=m}^{n-1}(s_{i+1} - s_i)| \\
                &\leq \sum_{i=m}^{n-1}|s_{i+1} - s_i| \\
                 &< \sum_{i=m}^{n-1}2^{-i} \\
                &= \sum_{i=1}^{n-1}2^{-i} - \sum_{i=1}^{m-1}2^{-i} \\
                &= \frac{1-2^{1-n}}{2(1-\frac{1}{2})} - \frac{1-2^{1-m}}{2(1-\frac{1}{2})} \\
                &=  2^{1-m}-2^{1-n} \\
                &= (2)2^{-m} < 2^{-N + 1}
            \end{align*}
        \end{proof}
        Now to prove the question.
        \begin{proof}
            Given $\epsilon > 0$ choose $N$ = $\log_2(\frac{1}{\epsilon}) + 1$
            
            Now let $m,n > N$, then by the lemma:
            \begin{align*}
                |s_{n} - s_{m}| &< 2^{-N+1} \\
                &= 2^{-(\log_2(\frac{1}{\epsilon}) + 1 ) + 1} \\
                &= 2^{\log_2(\epsilon)} \\
                &= \epsilon
            \end{align*}
            Therefore it is Cauchy and converges.
        \end{proof}
    \end{lemma}

\end{enumerate}
\end{document}
