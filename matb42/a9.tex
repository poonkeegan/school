\documentclass{article}
\usepackage[utf8]{inputenc}
\usepackage[margin=1in]{geometry}
\usepackage{amsmath, amsfonts, amsthm}
\usepackage{fancyhdr}
\usepackage{multicol}
\usepackage{graphicx}
\usepackage[inline]{enumitem}
\usepackage{wrapfig}
\usepackage[export]{adjustbox}
\graphicspath{ {images/} }
\pagestyle{empty}
\fancyhf{}
\cfoot{\thepage}
\pagenumbering{gobble}

\lhead{MATB42: Assignment \#8}
\rhead{
Poon, Keegan\\
1002423727\\
Mar 20th 2018}
\newcommand{\norm}[1]{\| #1 \|}
\newcommand{\deriv}[1]{\frac{d}{d #1}}
\newcommand{\parti}[1]{\frac{\partial}{\partial #1}}
\renewcommand{\headrulewidth}{0pt}
\newcommand{\gam}{\boldsymbol{\gamma}}
\newcommand{\divt}{\text{div}}
\begin{document}

\thispagestyle{fancy}
\begin{enumerate}
    \item 
    \begin{enumerate}
        \item Let $f, g: \mathbb{R}^n \rightarrow \mathbb{R}; \; \boldsymbol F,\, \boldsymbol G :\; \mathbb{R}^n \rightarrow \mathbb{R}^n;$ and define $\Delta$, the \textit{Laplacian}, by $\displaystyle \Delta f = \sum_{i=1}^n \frac{\partial^2 f}{\partial x_i^2}$.

        Verify the following identities
        \begin{enumerate}[label=(\roman*)]
            \item div$ (\boldsymbol F + \boldsymbol G) =$ div $\boldsymbol F + $ div $\boldsymbol G$.
            \[ \divt( \boldsymbol F + \boldsymbol G) = \sum_{i=1}^n \frac{\partial (F_i + G_i)}{\partial x_i} = \sum_{i=1}^n \frac{\partial F_i}{\partial x_i} + \frac{\partial G_i}{\partial x_i} = \sum_{i=1}^n \frac{\partial F_i}{\partial x_i} + \sum_{i=1}^n \frac{\partial G_i}{\partial x_i} = \divt \boldsymbol F + \divt \boldsymbol G\]
            \item div$ (f \boldsymbol F ) =f$ div $\boldsymbol F + \boldsymbol F \cdot$ grad$f$.

            \[\divt (f \boldsymbol F) = \sum_{i=1}^n \frac{\partial fF_i}{\partial x_i} = \sum_{i=1}^n \frac{\partial f}{\partial x_i} F_i + \frac{\partial F_i}{\partial x_i}f = \sum_{i=1}^n \frac{\partial f}{\partial x_i} F_i + f \sum_{i=1}^n\frac{\partial F_i}{\partial x_i} = \nabla f \cdot \boldsymbol F + f \divt \boldsymbol F \]
            \item $\Delta (f g) =f \Delta g + g \Delta f + 2($grad$f)\cdot ( $grad$ g).$
            \begin{proof}
                \begin{align*}
                    \Delta(fg) &= \sum_{i=1}^n\frac{\partial^2fg}{\partial^2 x_i } \\
                    &= \sum_{i=1}^n\frac{\partial}{\partial x_i } \bigg[ \frac{\partial f}{\partial x_i }g  + \frac{\partial g}{\partial x_i }f\bigg] \\
                    &= \sum_{i=1}^n \bigg[ \frac{\partial f}{\partial^2 x_i }g  + \bigg( \frac{\partial f}{\partial x_i } \cdot \frac{\partial g}{\partial x_i } \bigg) + \frac{\partial g}{\partial^2 x_i }f + \bigg( \frac{\partial f}{\partial x_i } \cdot \frac{\partial g}{\partial x_i } \bigg)\bigg] \\
                    &= g\sum_{i=1}^n \frac{\partial f}{\partial^2 x_i }  + 2 \sum_{i=1}^n \bigg( \frac{\partial f}{\partial x_i } \cdot \frac{\partial g}{\partial x_i } \bigg) + f\sum_{i=1}^n \frac{\partial g}{\partial^2 x_i } \\
                    &= g \Delta f  + 2[\nabla f \cdot \nabla g] + f \Delta g\\
                \end{align*}
            \end{proof}
        \end{enumerate}
        \item Let $f,g:D \subset \mathbb{R}^3 \rightarrow \mathbb{R}$ be of class $C^1$. If $R$ is a solid region contained in $D$ then
        \[\iiint_R \nabla f \cdot \nabla g \, dV = \iint_{\partial R} f \nabla  g \cdot \boldsymbol n \, dS - \iiint_R f \nabla^2 g \, dV\] 
        $(\nabla^2 g = $ div $(\nabla g))$.

        \begin{proof}
            \begin{align*}
                \iiint_R \nabla f \cdot \nabla g \, dV = \iint_{\partial R} f \nabla  g \cdot \boldsymbol n \, dS - \iiint_R f \nabla^2 g \, dV \\
               \iff \iiint_R \nabla f \cdot \nabla g \, dV + \iiint_R f \nabla^2 g \, dV = \iint_{\partial R} f \nabla  g \cdot \boldsymbol n \, dS \\
            \end{align*}
            \begin{align*}
                \iint_{\partial R} f \nabla  g \cdot \boldsymbol n \, dS \overset{\text{Div Thm}}{=} \iiint_R \text{div} (f \nabla g) \, dV  \overset{\text{(ii)}}{=} \iiint_R f(\text{div} \nabla g) + \nabla g \cdot \nabla f \, dV  \\
                = \iiint_R f(\text{div} \nabla g) \, dV + \iiint_R \nabla f \cdot \nabla g \, dV = \iiint_R f \nabla^2 g \, dV + \iiint_R \nabla f \cdot \nabla g \, dV \\
            \end{align*}
        \end{proof}
    \end{enumerate} 
    \newpage
    \item Use the Divergence Theorem to verify your answer to question 7 on assignment 8.
    \newpage
    \item Let $\boldsymbol F (x,y,z) = (x,y^2,e^{yz})$ and let $R$ be a cube centered at the origin with sides of length 2. Evaluate $\displaystyle \int_S$ div $\boldsymbol F \, dV $ directly and by using the Divergence Theorem. 

    Directly:
    \begin{align*}
        \int_S \divt \boldsymbol F \, dV &= \int_{-1}^1 \int_{-1}^1 \int_{-1}^1 1 + 2y + ye^{yz} \, dx \, dz \, dy \\
        &= \int_{-1}^1 \int_{-1}^1 2 + 4y + 2ye^{yz} \, dz \, dy \\
        &= \int_{-1}^1 4 + 8y + \big[e^{yz}\big]_{-1}^1\, dy \\
        &= \int_{-1}^1 4 + 8y + e^y - e^{-y} \, dy \\
        &= 8 + 0 + [e^y]_{-1}^1 + [e^{-y}]_{-1}^1 = 8
    \end{align*}
    Divergence:
        \[\int_R \divt \boldsymbol F \, dV = \int_S \boldsymbol F \cdot d\boldsymbol S\]
        where $S$ is the union of 6 planes with $-1 \leq x,y,z \leq 1$ and having normals in the $\pm x,y,z$ directions.
        The parameterizations can be given by
        \begin{align*}
            \boldsymbol \Phi_{S_1} (x,y) &= (x,y,1)\; &\boldsymbol \Phi_{S_1'} (x,y) &= (x,y,-1)\\
            \boldsymbol \Phi_{S_2} (y,z) &= (1,y,z)\; &\boldsymbol \Phi_{S_2'} (y,z) &= (-1,y,z)\\
            \boldsymbol \Phi_{S_3} (x,z) &= (x,1,z)\; &\boldsymbol \Phi_{S_3'} (x,z) &= (x,-1,z)\\
        \end{align*}
        When plugging in $S_3,S_3'$ into $\boldsymbol F$, then taking the dot with the normal, we get that the integrands are going to be the same, specifically 1. When doing the same with $S_2,S_2$, due to $x$ being odd, we get both being negative the other, so integrating over the same range will cancel each other out. Putting this together, the integral over $S$ is:
        \begin{align*}
            \int_S \boldsymbol F \cdot d\boldsymbol S &= 2 \int_{-1}^{1} \int_{-1}^{1} 1 \, dx \, dz + \int_{-1}^{1} \int_{-1}^{1} e^y \, dx \, dy + \int_{-1}^{1}\int_{-1}^{1} - e^{-y} \,dx\, dy \\
            &= 8 + 2\int_{-1}^{1} e^y \, dy - 2 \int_{-1}^{1} e^{-y} \, dy \\
            &= 8 + 2([e^y]_{-1}^1 + [e^{-y}]_{-1}^1) = 8
        \end{align*}
        \newpage
        \item Let $B$ be the pyramid with top vertex (0,0,1) and base vertices (0,0,0), (1,0,0), (0,1,0) and (1,1,0). Let $S$ be the 2-dim closed surface bounding $B$, oriented in the outward direction. Use Gauss' theorem to calculate $\displaystyle \int_S \boldsymbol F \cdot \, d \boldsymbol S$, where $\boldsymbol F(x,y,z) = (x^2y,3y^2z,9z^2x).$
            \[\int_R \divt \boldsymbol F \, dV = \int_S \boldsymbol F \cdot d\boldsymbol S\]
            \begin{align*}
                \divt \boldsymbol F &= 2xy + 6yz + 18zx \\
                \int_R \divt \boldsymbol F &= \int_0^1 \int_0^{1-z} \int_0^{1-z} 2xy + 6yz + 18zx \, dx \, dy \, dz \\
                &= \int_0^1 \int_0^{1-z} (1-z)^2y + 6yz(1-z) + 9z(1-z)^2 \, dy \, dz \\
                &= \int_0^1 (1/2)(1-z)^4 + 3z(1-z)^3 + 9z(1-z)^3 \, dz \\
                &= \int_0^1 -\frac{23z^4}{2} + 34z^3 - 33z^2 + 10z + \frac{1}{2} \, dz \\
                &= -\frac{23}{10} + \frac{34}{4} - 11 + 5 + \frac{1}{2} \\
                &= \frac{7}{10} \\ 
            \end{align*}
            \newpage
\item Use the Divergence Theorem to evaluate $\displaystyle \int_S \boldsymbol F \cdot \, d \boldsymbol S,$ where $\displaystyle \boldsymbol F (x,y,z) = \bigg(z^2 x, \frac{y^3}{3} + \tan z, x^2z+y^2 \bigg)$ and $S$ is the top half of the unit sphere $x^2 + y^2 + z^2 = 1$, oriented by the unit normal which points away from the origin.
    
    For the Divergence Theorem to apply, the integral must be over a closed, oriented outwards. To use Divergence Theorem the surface needs to be closed, so add the disk $S'$ on the $xy$-plane to the surface. Then
    \[\iiint_R \divt \boldsymbol F \, dV - \iint_{S'} \boldsymbol F \cdot d \boldsymbol S = \iint_S \boldsymbol F \cdot d\boldsymbol S\]
    The disk can be parametrized as $\boldsymbol \Phi (r, \theta) = (r\cos \theta, r\sin \theta, 0)$ with $r \in [0,1], \theta \in [0,2\pi]$
    \begin{align*} 
        \iint_{S'} \boldsymbol F \cdot d \boldsymbol S &= \int_0^1 \int_0^{2\pi} (*,*,(r\sin \theta)^2) \cdot (0,0,-1) \, d \theta \,d r
    \end{align*}

\item Let the electric field from a point source at the origin be given by $\displaystyle \boldsymbol E (\boldsymbol x) = \frac{\boldsymbol x}{\norm{\boldsymbol x}^3}$
    \begin{enumerate}
        \item What is the outward flux of $\boldsymbol E$ across the surface $\displaystyle \frac{x^2}{3} + \frac{2y^2}{5} + z^2 = 7$.
        \item Show that the flux of $\boldsymbol E$ across that part of the sphere $x^2 + y^2 + z^2 = 25$ with $z \geq 3$ is equal to the flux across that part of the plane $z = 3$ with $x^2 + y^2 \leq 16$.
    \end{enumerate} 
    \item Let $f : \mathbb{R}^n \rightarrow \mathbb{R}$ be given by $f(x,y,z) = x^2yz$ and let $\eta$ be the 2-form on $\mathbb{R}^3$ given by \[ \eta = (\sin x) \, dx \, dy + (e^y + xyz) \, dx \, dz + (x^2y^2) \, dy \, dz .\]
    \begin{enumerate}
        \item Compute $df$ and $d\eta$.
        \item Evaluate $df \wedge \eta$.
    \end{enumerate} 
\end{enumerate}
\end{document}
