\documentclass{article}
\usepackage[utf8]{inputenc}
\usepackage[margin=1in]{geometry}
\usepackage{amsmath, amsfonts}
\usepackage{fancyhdr}
\usepackage{multicol}
\usepackage{graphicx}
\usepackage[inline]{enumitem}
\usepackage{wrapfig}
\usepackage[export]{adjustbox}
\graphicspath{ {images/} }
\pagestyle{empty}
\fancyhf{}
\cfoot{\thepage}
\pagenumbering{gobble}

\lhead{MATB42: Assignment \#8}
\rhead{
Poon, Keegan\\
1002423727\\
Mar 20th 2018}
\newcommand{\norm}[1]{\| #1 \|}
\newcommand{\deriv}[1]{\frac{d}{d #1}}
\newcommand{\parti}[1]{\frac{\partial}{\partial #1}}
\renewcommand{\headrulewidth}{0pt}
\newcommand{\gam}{\boldsymbol{\gamma}}
\newcommand{\divt}{\text{div}}
\begin{document}

\thispagestyle{fancy}
\begin{enumerate}
    \item 
    \begin{enumerate}
        \item Let $f, g: \mathbb{R}^n \rightarrow \mathbb{R}; \; \boldsymbol F,\, \boldsymbol G :\; \mathbb{R}^n \rightarrow \mathbb{R};$ and define $\Delta$, the \textit{Laplacian}, by $\displaystyle \Delta f = \sum_{i=1}^n \frac{\partial^2 f}{\partial x_i^2}$.

        Verify the following identities
        \begin{enumerate}[label=(\roman*)]
            \item div$ (\boldsymbol F + \boldsymbol G) =$ div $\boldsymbol F + $ div $\boldsymbol G$.
            \[ \divt( \boldsymbol F + \boldsymbol G) = \sum_{i=1}^n \frac{\partial (\boldsymbol F + \boldsymbol G)}{\partial x_i} = \sum_{i=1}^n \frac{\partial \boldsymbol F)}{\partial x_i} + \frac{\partial \boldsymbol G)}{\partial x_i}\]
            \item div$ (f \boldsymbol F ) =f$ div $\boldsymbol F + \boldsymbol F \cdot$ grad$f$.
            \item $\Delta (f g) =f \Delta g + g \Delta f + 2($grad$f)\cdot ( $grad$ g).$
        \end{enumerate}
        \item Let $f,g:D \subset \mathbb{R}^3 \rightarrow \mathbb{R}$ be of class $C^1$. If $R$ is a solid region contained in $D$ then
        \[\iiint_R \nabla f \cdot \nabla g \, dV = \iint_{\partial R} f \nabla  g \cdot \boldsymbol n \, dS - \iiint_R f \nabla^2 g \, dV\] 
        $(\nabla^2 g = $ div $(\nabla g))$.
    \end{enumerate} 
    \item Use the Divergence Theorem to verify your asnwer to question 7 on assignment 8.
    \item Let $\boldsymbol F (x,y,z) = (x,y^2,e^{yz})$ and let $R$ be a cube centered at the origin with sides of length 2. Evaluate $\int_S$ div $\boldsymbol F \, dV $ directly and by using the Divergence Theorem.
        \item Let $B$ be the pyramid with top vertex (0,0,1) and base vertices (0,0,0), (1,0,0), (0,1,0) and (1,1,0). Let $S$ be the 2-dim closed surface bounding $B$, oriented in the outward direction. Use Gauss' theorem to calculate $\displaystyle \int_S \boldsymbol F \cdot \, d \boldsymbol S$, where $\boldsymbol F(x,y,z) = (x^2y,3y^2z,9z^2x).$

\item Use the Divergence Theorem to evaluate $\displaystyle \int_S \boldsymbol F \cdot \, d \boldsymbol S,$ where $\displaystyle \boldsymbol F (x,y,z) = \bigg(z^2 x, \frac{y^3}{3} + \tan z, x^2z+y^2 \bigg)$ and $S$ is the top half of the unit sphere $x^2 + y^2 + z^2 = 1$, oriented by the unit normalwhich points away from the origin.

\item Let the electric field from a point source at the origin be given by $\displaystyle \boldsymbol E (\boldsymbol x) = \frac{\boldsymbol x}{\norm{\boldsymbol x}^3}$
    \begin{enumerate}
        \item What is the outward flux of $\boldsymbol E$ across the surface $\displaystyle \frac{x^2}{3} + \frac{2y^2}{5} + z^2 = 7$.
        \item Show that the flux of $\boldsymbol E$ across that part of the sphere $x^2 + y^2 + z^2 = 25$ with $z \geq 3$ is equal to the flux across that part of the plane $z = 3$ with $x^2 + y^2 \leq 16$.
    \end{enumerate} 
    \item Let $f : \mathbb{R}^n \rightarrow \mathbb{R}$ be given by $f(x,y,z) = x^2yz$ and let $\eta$ be the 2-form on $\mathbb{R}^3$ given by \[ \eta = (\sin x) \, dx \, dy + (e^y + xyz) \, dx \, dz + (x^2y^2) \, dy \, dz .\]
    \begin{enumerate}
        \item Compute $df$ and $d\eta$.
        \item Evaluate $df \wedge \eta$.
    \end{enumerate} 
\end{enumerate}
\end{document}
