\documentclass{article}
\usepackage[utf8]{inputenc}
\usepackage[margin=1in]{geometry}
\usepackage{amsmath, amsfonts, amsthm}
\usepackage{fancyhdr}
\usepackage{multicol}
\usepackage{graphicx}
\usepackage[inline]{enumitem}
\usepackage{wrapfig}
\usepackage[export]{adjustbox}
\graphicspath{ {images/} }
\pagestyle{empty}
\fancyhf{}
\cfoot{\thepage}
\pagenumbering{gobble}

\lhead{MATB42: Assignment \#10}
\rhead{
Poon, Keegan\\
1002423727\\
Apr 3rd 2018}
\newcommand{\norm}[1]{\| #1 \|}
\newcommand{\deriv}[1]{\frac{d}{d #1}}
\newcommand{\parti}[1]{\frac{\partial}{\partial #1}}
\newcommand{\partis}[2]{\frac{\partial #2}{\partial #1}}
\renewcommand{\headrulewidth}{0pt}
\newcommand{\gam}{\boldsymbol{\gamma}}
\newcommand{\divt}{\text{div} \,}
\begin{document}

\thispagestyle{fancy}
\begin{enumerate}
    \item Let $\boldsymbol F$ be a vector field on $\mathbb{R}^3$ given by $\boldsymbol F = (F_1, F_2, F_3)$ where $F_1$, $F_2$, and $F_3$ are $C^1$-functions from $\mathbb{R}^3 \rightarrow \mathbb{R}$
    \begin{enumerate}
        \item Let $\eta$ be the 2-form given by
            \[ \eta = F_3 \, dx \, dy + F_1 \, dy \, dz + F_2 \, dz \, dx \]
            Show that $d\eta = (\divt \boldsymbol F) \, dx \, dy \, dz$
            
            (page 489, \#6)

            \begin{align*} 
                \eta &= F_3 \, dx \, dy + F_1 \, dy \, dz + F_2 \, dz \, dx \\
                d\eta &= d(F_3 \, dx \, dy + F_1 \, dy \, dz + F_2 \, dz \, dx) \\
                &= (dF_3)\, dx \, dy + (dF_1) \, dy \, dz + (dF_2) \, dz \, dx \\
                &= (\parti{x} F_3 \,dx + \parti{y} F_3 \, dy + \parti{z} F_3  \, dz)\, dx \, dy + (dF_1) \, dy \, dz + (dF_2) \, dz \, dx \\
                &= \parti{z} F_3  \, dz \, dx \, dy + (dF_1) \, dy \, dz + (dF_2) \, dz \, dx \\
                &= \parti{z} F_3  \, dx \, dy \, dz + (\parti{x} F_1 \,dx + \parti{y} F_1 \, dy + \parti{z} F_1  \, dz) \, dy \, dz + (dF_2) \, dz \, dx \\
                &= \parti{z} F_3  \, dx \, dy \, dz + \parti{x} F_1 \,dx \, dy \, dz + (dF_2) \, dz \, dx \\
                &= \parti{z} F_3  \, dx \, dy \, dz + \parti{x} F_1 \,dx \, dy \, dz + (\parti{x} F_2 \,dx + \parti{y} F_2 \, dy + \parti{z} F_2  \, dz) \, dz \, dx \\
                &= \parti{z} F_3  \, dx \, dy \, dz + \parti{x} F_1 \,dx \, dy \, dz + \parti{y} F_2 \, dy \, dz \, dx \\
                &= \parti{z} F_3  \, dx \, dy \, dz + \parti{x} F_1 \,dx \, dy \, dz + \parti{y} F_2 \, dx \, dy \, dz \\
                &= \parti{x} F_1 + \parti{y} F_2 + \parti{z} F_3 \, dx \, dy \, dz = (\divt \boldsymbol F) \, dx \, dy \, dz
            \end{align*} 
        \newpage
        \item Show that $dF_1 \wedge dF_2 \wedge dF_3 = (\text{det } D \boldsymbol F ) \, dx \, dy \, dz$
            \[ df = \sum_{i=0}^{n} \partis{x_i}{f} \, dx_i \]
            \begin{align*} 
                dF_1 \wedge dF_2 \wedge dF_3 &= (\partis{x}{F_1} \, dx + \partis{y}{F_1} \, dy + \partis{z}{F_1} \, dz) \wedge (\partis{x}{F_2} \, dx + \partis{y}{F_2} \, dy + \partis{z}{F_2} \, dz) \wedge dF_3 \\
                &= (\partis{x}{F_1} \, dx \wedge (\partis{x}{F_2} \, dx + \partis{y}{F_2} \, dy + \partis{z}{F_2} \, dz) \\
                &\; \; \; \; + \partis{y}{F_1} \, dy \wedge (\partis{x}{F_2} \, dx + \partis{y}{F_2} \, dy + \partis{z}{F_2} \, dz) \\
                &\; \; \; \; + \partis{z}{F_1} \, dz \wedge (\partis{x}{F_2} \, dx + \partis{y}{F_2} \, dy + \partis{z}{F_2} \, dz) ) \wedge dF_3 \\
                &= ((\partis{x}{F_1} \partis{y}{F_2} \, dx \, dy + \partis{x}{F_1} \partis{z}{F_2} \, dx \, dz) \\
                &\; \; \; \; + (\partis{y}{F_1} \partis{x}{F_2} \, dy \, dx + \partis{y}{F_1} \partis{z}{F_2} \, dy \, dz) \\
                &\; \; \; \; + (\partis{z}{F_1} \partis{x}{F_2} \, dz \, dx + \partis{z}{F_1} \partis{y}{F_2} \, dz \, dy))\wedge dF_3 \\
                &= ((\partis{x}{F_1} \partis{y}{F_2} - \partis{y}{F_1} \partis{x}{F_2}\, dx \, dy) \\
                &\; \; \; \; + (\partis{y}{F_1} \partis{z}{F_2} - \partis{z}{F_1} \partis{y}{F_2} \, dy \, dz) \\
                &\; \; \; \; + (\partis{z}{F_1} \partis{x}{F_2} - \partis{x}{F_1} \partis{z}{F_2} \, dz \, dx ))\wedge (\partis{x}{F_3} \, dx + \partis{y}{F_3} \, dy + \partis{z}{F_3} \, dz) \\
                &= (\partis{z}{F_3} (\partis{x}{F_1} \partis{y}{F_2} - \partis{y}{F_1} \partis{x}{F_2}\, dx \, dy \, dz) \\
                &\; \; \; \; + \partis{x}{F_3} (\partis{y}{F_1} \partis{z}{F_2} - \partis{z}{F_1} \partis{y}{F_2} \, dy \, dz \, dx) \\
                &\; \; \; \; + \partis{y}{F_3} (\partis{z}{F_1} \partis{x}{F_2} - \partis{x}{F_1} \partis{z}{F_2} \, dz \, dx \, dy)) \\
                &=  \partis{x}{F_3} (\partis{y}{F_1} \partis{z}{F_2} - \partis{z}{F_1} \partis{y}{F_2}) \, dx \, dy \, dz \\
                &\; \; \; \; - \partis{y}{F_3} (\partis{x}{F_1} \partis{z}{F_2} - \partis{z}{F_1} \partis{x}{F_2}) \, dx \, dy \, dz \\
                &\; \; \; \; + \partis{z}{F_3} (\partis{x}{F_1} \partis{y}{F_2} - \partis{y}{F_1} \partis{x}{F_2})\, dx \, dy \, dz \\
                &=  \partis{x}{F_3} \begin{vmatrix} \partis{y}{F_1} & \partis{z}{F_1} \\ \partis{y}{F_2} & \partis{z}{F_2} \end{vmatrix} 
                - \partis{y}{F_3} \begin{vmatrix} \partis{x}{F_1} & \partis{z}{F_1} \\ \partis{x}{F_2} & \partis{z}{F_2} \end{vmatrix}
                + \partis{z}{F_3} \begin{vmatrix} \partis{x}{F_1} & \partis{y}{F_1} \\ \partis{x}{F_2} & \partis{y}{F_2} \end{vmatrix} \, dx \, dy \, dz \\
                &=
                \begin{vmatrix}
                    \partis{x}{F_1} & \partis{y}{F_1} & \partis{z}{F_1} \\
                    \partis{x}{F_2} & \partis{y}{F_2} & \partis{z}{F_2} \\
                    \partis{x}{F_3} & \partis{y}{F_3} & \partis{z}{F_3} \\
                \end{vmatrix} \, dx \, dy \, dz
            \end{align*} 
    \end{enumerate}
    \newpage
    \item Let $\omega$ be a $k$-form and let $\eta$ be a $\ell$-form. Find $d(d\omega \wedge \eta - \omega \wedge d\eta)$.
        \begin{align*}
            d(d\omega \wedge \eta - \omega \wedge d\eta) &= d(d\omega \wedge \eta) - d(\omega \wedge d\eta) \\
            &= (d^2 \omega \wedge \eta + (-1)^{k+1} (d \omega \wedge d \eta)) - (d \omega \wedge d \eta + (-1)^k (\omega \wedge d^2 \eta)) \\
            &= (-1)^{k+1} d \omega \wedge d \eta - d \omega \wedge d \eta \\
            &= ((-1)^{k+1} - 1) d \omega \wedge d \eta \\
        \end{align*} 
    \newpage
    \item Determine if $\eta = y \, dx \, dy + xz \, dy \, dz - yz \, dz \, dx$ is exact. If $\eta$ is exact find a 1-form $\omega$ with $d\omega = \eta$.
    Check if $d\eta = \mathcal{O}$ to see if $\eta$ closed.

    (compare with page 461, \# 22)

    \begin{align*}
        d\eta &= d (y\, dx \, dy + xz \, dy \, dz - yz \, dz \, dx) \\
        &= (dy \, dx \, dy + d(xz)\wedge  dy \, dz - d(yz) \wedge dz \, dx) \\
        &= ((z \, dx + x \, dz)\wedge  dy \, dz - (z\, dy + y \, dz) \wedge dz \, dx) \\
        &= (z \, dx) \wedge dy \, dz - (z\, dy) \wedge dz \, dx \\
        &= z \, dx \, dy \, dz - z \, dx \, dy \, dz = \mathcal{O}\\
    \end{align*}
    Since the polynomials of $x$, $y$ and $z$ defined throughout $\mathbb{R}^3$ and $\eta$ closed, it is exact.
    \newpage
    \item
    Evaluate $\displaystyle \iint_S \omega$, where $\omega = z\, dx \, dy + x \, dy \, dz + y \, dz \, dx$ and $S$ is the unit sphere, directly and by the Divergence Theorem.

    (page 489, \#12)

    Directly: 
    
    Parametrize the sphere $S$ as 
    \[ \boldsymbol \Phi (\varphi, \theta) = ( \cos\theta \sin\varphi , \sin\theta \sin\varphi, \cos\varphi)\; \text{with }\theta \in [0,2\pi],\, \varphi \in [0,\pi] \]
    \begin{align*}
        \iint_S \omega &= \iint_{\boldsymbol \Phi} z \, dx \, dy + \iint_{\boldsymbol \Phi} x \, dy \, dz + \iint_{\boldsymbol \Phi} y \, dz \, dx \\
        &= \int_0^{2\pi} \int_0^{\pi} \cos \varphi \begin{vmatrix}\partis{\varphi}{\cos\theta \sin \varphi} & \partis{\theta}{\cos\theta \sin \varphi} \\ \partis{\varphi}{\sin\theta \sin\varphi} & \partis{\theta}{\sin\theta \sin\varphi} \end{vmatrix} \, d\varphi \, d\theta + \int_0^{2\pi} \int_0^{\pi} \cos\theta \sin\varphi \begin{vmatrix} \partis{\varphi}{\sin\theta \sin\varphi} & \partis{\theta}{\sin\theta \sin\varphi} \\ \partis{\varphi}{\cos \varphi} & \partis{\theta}{\cos \varphi} \end{vmatrix}\, d\varphi \, d\theta \\
        & \; \; \; \; + \int_0^{2\pi} \int_0^{\pi} \sin\theta \sin\varphi \begin{vmatrix} \partis{\varphi}{\cos \varphi} & \partis{\theta}{\cos \varphi} \\ \partis{\varphi}{\cos\theta \sin\varphi} & \partis{\theta}{\cos\theta \sin\varphi}\end{vmatrix}\, d\varphi \, d\theta \\
        &= \int_0^{2\pi} \int_0^{\pi} \cos \varphi \begin{vmatrix}\cos \theta \cos \varphi & - \sin \theta \sin \varphi  \\ \sin \theta \cos \varphi & \cos \theta \sin \varphi \end{vmatrix} \, d\varphi \, d\theta + \int_0^{2\pi} \int_0^{\pi} \cos\theta \sin\varphi \begin{vmatrix} \sin \theta \cos \varphi & \cos \theta \sin \varphi \\ -\sin \varphi & 0 \end{vmatrix}\, d\varphi \, d\theta \\
        & \; \; \; \; + \int_0^{2\pi} \int_0^{\pi} \sin\theta \sin\varphi \begin{vmatrix} -\sin \varphi & 0 \\ \cos \theta \cos \varphi & - \sin \theta \sin \varphi \end{vmatrix}\, d\varphi \, d\theta \\
        &= \int_0^{2\pi} \int_0^{\pi} \sin \varphi \cos^2 \varphi \, d\varphi \, d\theta + \int_0^{2\pi} \int_0^{\pi} \cos^2\theta \sin^3\varphi + \sin^2\theta \sin^3\varphi \, d\varphi \, d\theta \\
        &= \int_0^{2\pi} \int_0^{\pi} \sin \varphi (\cos^2 \varphi + \sin^2 \varphi) \, d\varphi \, d\theta  \\
        &= \int_0^{2\pi} \int_0^{\pi} \sin \varphi  \, d\varphi \, d\theta  \\
        &= 2\pi \bigg[ -\cos \varphi\bigg]_0^{\pi} = 2\pi \\
    \end{align*}
    Divergence Theorem:
    \[ d\omega = dz \, dy \, dx + dx \, dy \, dz + dy dz dx = 3 \, dx \, dy \, dz \]
    \begin{align*}
        \iint_S \omega &= \iiint_R d \omega \\
        &= 3 \int_0^{2\pi} \int_0^{\pi} \int_0^1 \rho^2 \sin(\varphi) \, d\rho \, d\varphi \, d\theta \\
        &= \int_0^{2\pi} \int_0^{\pi} \sin(\varphi) \, d\varphi \, d\theta \\
        &= 2\pi \bigg[ -\cos \varphi\bigg]_0^{\pi} = 2\pi \\
    \end{align*}

    \newpage
    \item Compute $\displaystyle \int_S \omega$ and use symbolic algebra software to sketch $S$ in each of the following.
    \begin{enumerate}
        \item $\omega = xz \, dx \, dy + x^2 \, dy \, dz + dy \, dz \, dx$
  
        S is the upper hemisphere $x^2 + y^2 + z^2 = 4$, $z \geq 0$ with $\boldsymbol n$ pointing upward.
        
        Close it with the disk of radius 2 on the $xy$-plane to apply divergence theorem
        \[ \boldsymbol \Phi(\theta, r) = (r \cos \theta, r \sin \theta, 0),\; r \in [0,2],\, \theta \in [0, 2\pi] \]
        \begin{align*}
            dx\, dy &= \begin{vmatrix} -r\sin \theta & \cos \theta \\ r \cos \theta &  \sin \theta \end{vmatrix} = -2r \\
            &\text{ Which is negative, so correct orientation for normal pointing down.} \\
            dy\, dz &= 0 \text{ Since $z$ is 0} \\
            dz\, dx &= 0 
            \overset{\text{Div Thm}}{\implies}
        \end{align*}

        \item $\omega = z \, dx \, dy + x \, dy \, dz + y \, dz \, dx$

        $S$ is the part of the plane $x+y+z = 1$ which lies in the first octant oriented by the unit normal which points upward.

        \item $\omega  = xz \, dx \, dy + y \, dx \, dz + z^2 \, dy \, dz$

        $S$ is the part of the cone $z = \sqrt{x^2 + y^2}$ between $z=1$ and $z=3$,  oriented by the unit normal with negative $z$-component.

        \item $\omega = z \, dx \, dy + y \, dy \, dz$

        $S$ is the oriented surface given by the parametrization

        $\Phi (u,v) = (u+v, uv^2, u^2 + v^2),\, 0 \leq u \leq 1,\, 0\leq v \leq 1$.
    \end{enumerate}

    \newpage
    \item Verify Stokes' theorem by direct calculation of both sides when the surface $S$ is the piece of the paraboloid $z= x^2 + y^2 - 4$ with $z \leq 0$, oriented by the downward pointing unit normal, and $\omega = (2y-z)\, dx + (x + y^2 - z)\, dy + (4y-3x)\, dz$.

    As part of your solution, provide a sketch showing the appropriate orientations. (For this question you may draw the sketch by hand or use symbolic algebra software.)
    \newpage
    \item Let $\omega = yz\, dx - xz \, dy + xy \, dz$ and let $\boldsymbol \gamma(t) = (2\cos t, 2\sin t, 4),\, 0 \leq t \leq 2\pi$.
    \begin{enumerate}
        \item Let $S$ be the piece of the surface $z = x^2 + y^2$ with $z \leq 4$. Use Stokes' theorem to give an integral over $S$ which is equivalent to $\displaystyle \int_{\boldsymbol \gamma} \omega$. Verify by directly computing both integrals.
        \item Let $S'$ be the part of the plane $z=4$ with $x^2 + y^2 \leq 4$. Use Stokes' theorem to give an integral over $S'$ which is equivalent to $\displaystyle \int_{\gamma} \omega$. Verify by direct computation.
        
        \item Can you give another explanation as to why the integrals you get over $S$ and $S'$ should have the same value?

    \end{enumerate}
    
    \newpage
    \item Let $\boldsymbol F (x,y,z) = (e^{z^2}, 4z-y, 8x \sin y)$. Find $\displaystyle \int_S (\nabla \times \boldsymbol F)\cdot d\boldsymbol S$ where $S$ is the unit sphere oriented with the outward normal.

    \newpage
    \item 
    \begin{enumerate}
        \item Marsden \& Tromba, page 451, \# 13.
        \item Marsden \& Tromba, page 451, \# 15.
        \item Use symbolic algebra software to sketch the surfaces in parts (a) and (b).
    \end{enumerate}
    
    \newpage
    \item 
    \begin{enumerate}
        \item Let $\boldsymbol F$ and $\boldsymbol G$ be vector fields on $\mathbb{R}^3$ and let $f : \mathbb{R}^3 \rightarrow \mathbb{R}$. Verify the following identities.
        \begin{enumerate}[label=(\roman*)]
            \item $\divt (\boldsymbol F \times \boldsymbol G) = \boldsymbol G \cdot \text{curl} \, \boldsymbol F - \boldsymbol F \cdot \text{curl} \, \boldsymbol G.$
            \item curl$(f \boldsymbol F) = f \text{curl} \boldsymbol F + (\text{grad}\, f) \times \boldsymbol F$.
        \end{enumerate}
        \item Let $R$ be a closed region in $\mathbb{R}^3$ with boundary $\partial R$. Prove the identity
        \[\int_{\partial R} ( \boldsymbol F \times \text{curl} \boldsymbol G) \cdot \, d \boldsymbol S = \int_R(\text{curl} \boldsymbol F ) \cdot (\text{curl} \boldsymbol G) \, dV - \int_R \boldsymbol F \cdot \text{curl}(\text{curl} \boldsymbol G ) \, dV\]
        (page 490, \#2)
    \end{enumerate}
\end{enumerate}
\end{document}
