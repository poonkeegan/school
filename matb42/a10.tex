\documentclass{article}
\usepackage[utf8]{inputenc}
\usepackage[margin=1in]{geometry}
\usepackage{amsmath, amsfonts, amsthm}
\usepackage{fancyhdr}
\usepackage{multicol}
\usepackage{graphicx}
\usepackage[inline]{enumitem}
\usepackage{wrapfig}
\usepackage[export]{adjustbox}
\graphicspath{ {images/} }
\pagestyle{empty}
\fancyhf{}
\cfoot{\thepage}
\pagenumbering{gobble}

\lhead{MATB42: Assignment \#10}
\rhead{
Poon, Keegan\\
1002423727\\
Apr 3rd 2018}
\newcommand{\norm}[1]{\| #1 \|}
\newcommand{\deriv}[1]{\frac{d}{d #1}}
\newcommand{\parti}[1]{\frac{\partial}{\partial #1}}
\renewcommand{\headrulewidth}{0pt}
\newcommand{\gam}{\boldsymbol{\gamma}}
\newcommand{\divt}{\text{div} \,}
\begin{document}

\thispagestyle{fancy}
\begin{enumerate}
    \item Let $\boldsymbol F$ be a vector field on $\mathbb{R}^3$ given by $\boldsymbol F = (F_1, F_2, F_3)$ where $F_1$, $F_2$, and $F_3$ are $C^1$-functions from $\mathbb{R}^3 \rightarrow \mathbb{R}$
    \begin{enumerate}
        \item Let $\eta$ be the 2-form given by
            \[ \eta = F_3 \, dx \, dy + F_1 \, dy \, dz + F_2 \, dz \, dx \]
            Show that $d\eta = (\divt \boldsymbol F) \, dx \, dy \, dz$
            
            (page 489, \# 6)
    \end{enumerate}
\end{enumerate}
\end{document}
