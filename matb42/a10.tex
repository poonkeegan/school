\documentclass{article}
\usepackage[utf8]{inputenc}
\usepackage[margin=1in]{geometry}
\usepackage{amsmath, amsfonts, amsthm}
\usepackage{fancyhdr}
\usepackage{multicol}
\usepackage{graphicx}
\usepackage[inline]{enumitem}
\usepackage{wrapfig}
\usepackage[export]{adjustbox}
\graphicspath{ {images/} }
\pagestyle{empty}
\fancyhf{}
\cfoot{\thepage}
\pagenumbering{gobble}

\lhead{MATB42: Assignment \#10}
\rhead{
Poon, Keegan\\
1002423727\\
Apr 3rd 2018}
\newcommand{\norm}[1]{\| #1 \|}
\newcommand{\deriv}[1]{\frac{d}{d #1}}
\newcommand{\parti}[1]{\frac{\partial}{\partial #1}}
\renewcommand{\headrulewidth}{0pt}
\newcommand{\gam}{\boldsymbol{\gamma}}
\newcommand{\divt}{\text{div} \,}
\begin{document}

\thispagestyle{fancy}
\begin{enumerate}
    \item Let $\boldsymbol F$ be a vector field on $\mathbb{R}^3$ given by $\boldsymbol F = (F_1, F_2, F_3)$ where $F_1$, $F_2$, and $F_3$ are $C^1$-functions from $\mathbb{R}^3 \rightarrow \mathbb{R}$
    \begin{enumerate}
        \item Let $\eta$ be the 2-form given by
            \[ \eta = F_3 \, dx \, dy + F_1 \, dy \, dz + F_2 \, dz \, dx \]
            Show that $d\eta = (\divt \boldsymbol F) \, dx \, dy \, dz$
            
            (page 489, \#6)

            \begin{align*} 
                \eta &= F_3 \, dx \, dy + F_1 \, dy \, dz + F_2 \, dz \, dx \\
                d\eta &= d(F_3 \, dx \, dy + F_1 \, dy \, dz + F_2 \, dz \, dx) \\
                &= (dF_3)\, dx \, dy + (dF_1) \, dy \, dz + (dF_2) \, dz \, dx \\
                &= (\parti{x} F_3 \,dx + \parti{y} F_3 \, dy + \parti{z} F_3  \, dz)\, dx \, dy + (dF_1) \, dy \, dz + (dF_2) \, dz \, dx \\
                &= \parti{z} F_3  \, dz \, dx \, dy + (dF_1) \, dy \, dz + (dF_2) \, dz \, dx \\
                &= \parti{z} F_3  \, dx \, dy \, dz + (\parti{x} F_1 \,dx + \parti{y} F_1 \, dy + \parti{z} F_1  \, dz) \, dy \, dz + (dF_2) \, dz \, dx \\
                &= \parti{z} F_3  \, dx \, dy \, dz + \parti{x} F_1 \,dx \, dy \, dz + (dF_2) \, dz \, dx \\
                &= \parti{z} F_3  \, dx \, dy \, dz + \parti{x} F_1 \,dx \, dy \, dz + (\parti{x} F_2 \,dx + \parti{y} F_2 \, dy + \parti{z} F_2  \, dz) \, dz \, dx \\
                &= \parti{z} F_3  \, dx \, dy \, dz + \parti{x} F_1 \,dx \, dy \, dz + \parti{y} F_2 \, dy \, dz \, dx \\
                &= \parti{z} F_3  \, dx \, dy \, dz + \parti{x} F_1 \,dx \, dy \, dz + \parti{y} F_2 \, dx \, dy \, dz \\
                &= \parti{x} F_1 + \parti{y} F_2 + \parti{z} F_3 \, dx \, dy \, dz = (\divt \boldsymbol F) \, dx \, dy \, dz
            \end{align*} 
        \item Show that $dF_1 \wedge dF_2 \wedge dF_3 = (\text{det } D \boldsymbol F ) \, dx \, dy \, dz$
    \end{enumerate}
    \newpage
    \item Let $\omega$ be a $k$-form and let $\eta$ be a $\ell$-form. Find $d(d\omega \wedge \eta - \omega \wedge d\eta)$.
    \newpage
    \item Determine if $\eta = y \, dx \, dy + dz \, dy \, dz - yz \, dz \, dx$ is exact. If $\eta$ is exact find a 1-form $\omega$ with $d\omega = \eta$.

    (compare with page 461, \# 22)
    \newpage
    \item
    Evaluate $\displaystyle \iint_S \omega$, where $\omega = z\, dx \, dy + x \, dy \, dz + y \, dz \, dx$ and $S$ is the unit sphere, directly and by the Divergence Theorem.

    (page 489, \#12)

    \newpage
    \item Compute $\displaystyle \int_S \omega$ and use symbolic algebra software to sketch $S$ in each of the following.
    \begin{enumerate}
        \item $\omega = xz \, dx \, dy + x^2 \, dy \, dz + dy \, dz \, dx$
        
        S is the upper hemisphere $x^2 + y^2 + z^2 = 4$, $z \geq 0$ with $\boldsymbol n$ pointing upward.


        \item $\omega = z \, dx \, dy + x \, dy \, dz + y \, dz \, dx$

        $S$ is the part of the plane $x+y+z = 1$ which lies in the first octant oriented by the unit normal which points upward.

        \item $\omega  = xz \, dx \, dy + y \, dx \, dz + z^2 \, dy \, dz$

        $S$ is the part of the cone $z = \sqrt{x^2 + y^2}$ between $z=1$ and $z=3$,  oriented by the unit normal with negative $z$-component.

        \item $\omega = z \, dx \, dy + y \, dy \, dz$
        
        $S$ is the oriented surface given by the parametrization

        $\boldsymbol \Phi (u,v) = (u+v, uv^2, y^2 + v^2), 0 \leq u \leq 1,\, 0\leq v \leq 1$.
    \end{enumerate}
\end{enumerate}
\end{document}
