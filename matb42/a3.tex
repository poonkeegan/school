\documentclass{article}
\usepackage[utf8]{inputenc}
\usepackage[margin=1in]{geometry}
\usepackage{amsmath, amsfonts}
\usepackage{fancyhdr}
\usepackage{multicol}
\usepackage{graphicx}
\graphicspath{ {images/} }
\pagestyle{empty}
\fancyhf{}
\cfoot{\thepage}

\lhead{MATB42: Assignment \#3 \\
Bonus}
\rhead{
Poon, Keegan\\
1002423727\\
Feb 6th 2018}
\newcommand{\norm}[1]{\| #1 \|}
\renewcommand{\headrulewidth}{0pt}
\begin{document}

\thispagestyle{fancy}

\begin{enumerate}
\item A particle moving on the curve $\boldsymbol{\gamma}(t) = (3t^2, -\sin t, -e^t)$ is released at time $t=\frac{1}{2}$ and flies off on a tangent. What are its coordinates at time $t=1$. 

Calculate the direction of the tangent.
\begin{align*} 
    \boldsymbol{\gamma}'(t) &= (3(2t), -\cos(t), -e^t) = (6t,-\cos(t), -e^t) \\
    \boldsymbol{\gamma}'\Big(\frac{1}{2}\Big) &= \Big(3, -\cos\Big(\frac{1}{2}\Big), -\sqrt{e}\Big)\\
\end{align*}

Calculate the position of the particle at $t=\frac{1}{2}$.
\begin{align*} 
    \boldsymbol{\gamma}\Big(\frac{1}{2}\Big) &= \Big(3\Big(\frac{1}{4}\Big), -\sin\Big(\frac{1}{2}\Big), -\sqrt{e}\Big) = \Big(\Big(\frac{3}{4}\Big), -\sin\Big(\frac{1}{2}\Big), -\sqrt{e}\Big)\\
\end{align*}

This means the particle should be at position $\displaystyle \boldsymbol{\gamma} \Big(\frac{1}{2}\Big) + \frac{1}{2} \boldsymbol{\gamma}' \Big( \frac{1}{2} \Big)$ which is equal to:

\begin{align*} 
    & \Big(\Big(\frac{3}{4}\Big), -\sin\Big(\frac{1}{2}\Big), -\sqrt{e}\Big) + \frac{1}{2}\Big(3, -\cos\Big(\frac{1}{2}\Big), -\sqrt{e}\Big) \\
    &= \Big(\frac{9}{4}, -\frac{ 2 \sin(\frac{1}{2}) + \cos(\frac{1}{2})}{2},-\frac{3\sqrt{e}}{2}\Big)
\end{align*}
\newpage
\item The displacement at time $t$ and horizontal position on a line $x$ of a certain violin string is given by $u=\sin(x-6t) + \sin(x+6t)$. Calculate the velocity of the string at $x=1$ when $t=\frac{1}{3}$.

At time $t=\frac{1}{3}$, the displacement function is given by
\begin{align*} 
    u = \sin(x-3) + \sin(x+3)
\end{align*}
Taking the rate of change we get
\begin{align*} 
    \frac{\partial u}{\partial x} = \cos(x-3) + \cos(x+3)
\end{align*}
So the velocity is $\cos(-2) + \cos(4)$
\newpage
\item Find paths $\boldsymbol{\gamma}(t)$ which represent
\begin{enumerate}
    \item the straight line segment in $\mathbb{R}$ from (1,2,3) to (4,-5,6).

        We want $\boldsymbol{\gamma}_1$ to range over 1 to 4, so we can take $\boldsymbol{\gamma}_1(t)= t+1$ over $0\leq t \leq 3$

        We want $\boldsymbol{\gamma}_2$ to range over 2 to -5, so we can take $\boldsymbol{\gamma}_1(t)= -\frac{7t}{3}+2$ over $0\leq t \leq 3$ to get $-7(3)/3 + 2 = -7 + 2 = -5$.

        We want $\boldsymbol{\gamma}_3$ to range over 3 to 6, so we can take $\boldsymbol{\gamma}_1(t)= t+3$ over $0\leq t \leq 3$
        \[ \boldsymbol{\gamma}(t) = \Big(t+1,2-\frac{7t}{3}, t+3 \Big),\: 0 \leq t \leq 3 \]
    \item the curve $\{(x,y)\in \mathbb{R}^2|3x^2+25y^2=4\}$
    
    This is an ellipse, so $x,y$ should be sin cos, exploiting the fact that $\sin^2 \alpha + \cos^2 \alpha = 1$ take the coefficients of the trig functions so that they cancel the existing ones. $\frac{1}{\sqrt{3}}$ and $\frac{1}{5}$ respectively. But this means the equation equals to 1, to make it 4, multiply each $x,y$ by $\sqrt{4}$ so when it is squared, a factor of four comes out.
        \[ \boldsymbol{\gamma}(t) = \Big(\frac{2}{\sqrt{3}} \sin(t), \frac{2}{5}\cos(t)\Big), \: 0 \leq t \leq 2\pi \]
    \item the curve of intersection of the cone $z=\sqrt{x^2 + y^2}$ and the sphere $x^2+y^2+z^2 = 2x$.
        
        To actually get the intersection, equate the two parameterizations plugging z of the cone into the sphere.
        \begin{align*}
            x^2 + y^2 + x^2 + y^2 &= 2x \\
            2x^2 - 2x +2y^2 &= 0 \\
            2(x^2 - x + \frac{1}{4}) - \frac{1}{2} +2y^2 &= 0 \\
            2(x^2 - x + \frac{1}{4}) +2y^2 &= \frac{1}{2} \\
            2(x - \frac{1}{2})^2 +2y^2 &= \frac{1}{2}
        \end{align*}
        From this ellipse equation, looking at the shift of the center of the ellipse and the constant factors, $x,y$ should be set to $\frac{\sin(t) + 1}{2}, \frac{\cos(t)}{2}$. Now to figure out $z$, just fit this into one of the earlier equations.
        \begin{align*}
            \Big( \frac{\sin(t) + 1}{2} \Big)^2 + \Big( \frac{\cos(t)}{2} \Big)^2 + z^2 &= \sin(t) + 1 \\
            \frac{\sin(t)^2 + 2\sin(t) + 1}{4} + \frac{\cos^2(t)}{4} + z^2 &= \sin(t) + 1 \\
            \frac{1}{4} + \frac{1}{4} + \frac{\sin(t)}{2} + z^2 &= \sin(t) + 1 \\
            z^2 &= \sqrt{\frac{\sin(t) + 1}{2}}
        \end{align*}
        \[ \boldsymbol{\gamma}(t) = \Big( \frac{\sin(t) + 1}{2}, \frac{\cos(t)}{2}, \frac{\sin(t) + 1}{2} \Big), \: 0 \leq t \leq 2\pi \]
\end{enumerate}

\newpage
\item Find the arclength of the following curves.
\begin{enumerate}
    \item $\displaystyle \boldsymbol{\gamma}(t) = \Bigg(t,\frac{1}{\sqrt{2}}t^2,\frac{1}{3}t^3 \Bigg)$ between the origin and $\displaystyle \Bigg(2,2\sqrt{2},\frac{8}{3}\Bigg)$.
    \begin{multicols}{2} 
    \noindent
    \begin{align*} 
        \boldsymbol{\gamma}'(t) &= \Big(1, \sqrt{2}t, t^2 \Big) \\
        \norm{\boldsymbol{\gamma}'(t)} &= \sqrt{1 + 2t^2 + t^4 } \\
        &= \sqrt{(t^2 + 1)^2 } \\ 
        &= (t^2 + 1) \
    \end{align*}
    \begin{align*}
        \int_{\boldsymbol{\gamma}} ds &= \int_0^2 \norm{\boldsymbol{\gamma}'(t)} dt \\
        &= \int_0^2 t^2 dt + \int_0^2 1 dt \\
        &= \frac{8}{3} + 2 \\
        &= 4 + \frac{2}{3}
    \end{align*}

    \end{multicols}
    \item $\displaystyle \boldsymbol{\gamma}(t) = (\cos 3t ,\sin 3t ,2t^{\frac{3}{2}}),$ $0 \leq t \leq 2$.
    \begin{align*} 
        \boldsymbol{\gamma}'(t) &= (-3\sin3t,3\cos3t,3t^\frac{1}{2}) \\
        \norm{\boldsymbol{\gamma}'(t)} &= 3\sqrt{\sin^23t+\cos^23t+t} \\
        &= 3\sqrt{t + 1} \\
    \end{align*}
    \begin{multicols}{2} 
    \noindent
    \begin{align*}
        \int_{\boldsymbol{\gamma}} ds &= \int_0^2 \norm{\boldsymbol{\gamma}'(t)} dt \\
        &= 3\int_0^2 \sqrt{t+1} dt \\
        &\text{Let } u = t + 1,\: du = dt \\
        &= 3\int_1^3 \sqrt{u} du \\
    \end{align*}
    \begin{align*}
        &= 2\Big[u^\frac{3}{2}\Big]_1^3 \\
        &= 2\sqrt{27} - 2 \\
        &= 6\sqrt{3} - 2 \\
    \end{align*}
    \end{multicols}
    \item $\displaystyle \boldsymbol{\gamma}(t) = (t^2, \sin t - t\cos t, \cos t + t\sin t),$ $0 \leq t \leq \pi$
    \begin{multicols}{2} 
    \noindent
    \begin{align*} 
        \boldsymbol{\gamma}'(t) &= (2t,\cos t-\cos t+t\sin t, -\sin t+\sin t+t\cos t) \\
        &= (2t,t\sin t,t\cos t) \\
        \norm{\boldsymbol{\gamma}'(t)} &= t\sqrt{4 + \sin^2t+\cos^2t} \\
        &= \sqrt{5}t \\
    \end{align*}
    \begin{align*}
        \int_{\boldsymbol{\gamma}} ds &= \int_0^\pi \norm{\boldsymbol{\gamma}'(t)} dt \\
        &= \sqrt{5}\int_0^\pi t dt \\
        &= \frac{\sqrt{5}\pi^2}{2}
    \end{align*}
    \end{multicols}
\end{enumerate}
\newpage
\item Evaluate the path integral $\displaystyle \int_{\boldsymbol{\gamma}}f(x,y,z)ds$, where
\begin{enumerate}
    \item $f(x,y,z) = x^2 - y + 3z$ and $\boldsymbol{\gamma}$ is a parameterization of the line segment from the origin to (1,2,1).
    \begin{multicols}{2} 
    \noindent
    \begin{align*} 
        \boldsymbol{\gamma}(t) &= (t,2t,t) \text{ from } 0 \leq t \leq 1 \\
        \boldsymbol{\gamma}'(t) &= (1, 2, 1) \\
        \norm{\boldsymbol{\gamma}'(t)} &= \sqrt{6} \\
    \end{align*}
    \begin{align*} 
        \int_{\boldsymbol{\gamma}}f(x,y,z)ds &= \int_0^1 f(\boldsymbol{\gamma}(t)) \norm{\boldsymbol{\gamma}'(t)}dt \\
        &= \sqrt{6} \int_0^1 t^2 - 2t + 3 t dt = \sqrt{6} \int_0^1 t^2 + t dt \\
        &= \sqrt{6} \Big(\frac{1}{3} + \frac{1}{2}\Big) \\
        &= \frac{5}{\sqrt{6}} \\
    \end{align*}
    \end{multicols}
    \item $f(x,y,z) = xyz$ and $\boldsymbol{\gamma}(t) = (-\sin 2t, \sqrt{2} \cos 2t, \sin 2t)$, $\displaystyle 0 \leq t \leq \frac{\pi}{4}$
    \begin{align*} 
        \boldsymbol{\gamma}'(t) &= (-2\cos 2t, -2\sqrt2\sin 2t, 2\cos 2t) \\
        \norm{\boldsymbol{\gamma}'(t)} &= \sqrt{4\cos^22t+8\sin^22t+4\cos^22t} \\
        &= \sqrt{8} = 2\sqrt{2}
    \end{align*}
    \begin{multicols}{2} 
    \noindent
    \begin{align*} 
        \int_{\boldsymbol{\gamma}}f(x,y,z)ds &= \int_0^{\frac{\pi}{4}} f(\boldsymbol{\gamma}(t)) \norm{\boldsymbol{\gamma}'(t)}dt \\
        &= -2\sqrt{2}\sqrt{2} \int_0^{\frac{\pi}{4}} \sin^22t \cos 2t dt = \\
        &\text{Let } u = \sin(2t),\: du = 2\cos(2t)dt \\
        &= -2 \int_0^{\sin(\frac{\pi}{4})} u^2 dt \\
    \end{align*}
    \begin{align*} 
        &= -\frac{2}{3} \Big[\sin(2t)^3\Big]_0^{\frac{\pi}{4}} \\
        &= -\frac{2}{3} \sin\Big(\frac{\pi}{2}\Big)^3 \\
        &= -\frac{2}{3} \\
    \end{align*}
    \end{multicols}
    
    \item $f(x,y,z) = xyz$ and $\boldsymbol{\gamma}(t) = (t,2t,3t)$, $0\leq t \leq 2$.
    \begin{multicols}{2} 
    \noindent
    \begin{align*} 
        \boldsymbol{\gamma}'(t) &= (1,2,3) \\
        \norm{\boldsymbol{\gamma}'(t)} &= \sqrt{1 + 4 + 9} \\
        &= \sqrt{14}
    \end{align*}
    \begin{align*} 
        \int_{\boldsymbol{\gamma}}f(x,y,z)ds &= \int_0^2 f(\boldsymbol{\gamma}(t)) \norm{\boldsymbol{\gamma}'(t)}dt \\
        &= 6\sqrt{14} \int_0^2 t^3 dt\\
        &= \frac{6}{4}\sqrt{14}(2^4) \\ 
        &= 24\sqrt{14}
    \end{align*}
    \end{multicols}

    \item $f(x,y,z) = 3x + xy + z^3$ and $\boldsymbol{\gamma}(t) = ( \cos 4t, \sin 4t, 3t)$, $0 \leq t \leq 2\pi$.
    \begin{align*} 
        \boldsymbol{\gamma}'(t) &= (-4 \sin 4t, 4 \cos 4t, 3) \\
        \norm{\boldsymbol{\gamma}'(t)} &= \sqrt{16( \sin^24t + \cos^24t) + 9} \\
        &= \sqrt{25} \\
        &= 5 \\
    \end{align*}
    \begin{multicols}{2} 
    \noindent
    \begin{align*} 
        \int_{\boldsymbol{\gamma}}f(x,y,z)ds &= \int_0^{2\pi} f(\boldsymbol{\gamma}(t)) \norm{\boldsymbol{\gamma}'(t)}dt \\
        &= 5 \int_0^{2\pi} 3\cos 4t + \sin 4t \cos 4t + 27t^3 dt\\
        &= 5 \int_0^{2\pi} (3 + \sin 4t) \cos 4t dt \\ 
        &\: +5 \int_0^{2\pi} + 27t^3 dt\\
    \end{align*}
    \begin{align*} 
        &\text{Let } u = 3 + \sin(4t),\: du = 4\cos(4t)dt \\
        &= \frac{5}{4} \int_3^{3 + \sin(8\pi)} (3 + u) du + \frac{5(27)}{4}\Big[t^4\Big]_0^{2\pi}\\
        &= \frac{5}{4} \int_3^{3} (3 + u) du + \frac{5(27)}{4}\Big[t^4\Big]_0^{2\pi}\\
        &= 20(27)\pi^4 \\
        &= 540\pi^4 \\
    \end{align*}
    \end{multicols}
    
    \item $f(x,y,z) = z \cos y$ and $\boldsymbol{\gamma}(t) = (-2t,3t,t)$, $0\leq t \leq 2$.
    \begin{multicols}{2} 
    \noindent
    \begin{align*} 
        \boldsymbol{\gamma}'(t) &= (-2,3,1) \\
        \norm{\boldsymbol{\gamma}'(t)} &= \sqrt{4 + 9 + 1} \\
        &= \sqrt{14} \\
    \end{align*}
    \begin{align*} 
        \int_{\boldsymbol{\gamma}}f(x,y,z)ds &= \int_0^2 f(\boldsymbol{\gamma}(t)) \norm{\boldsymbol{\gamma}'(t)}dt \\
        &= \sqrt{14} \int_0^2 t \cos 3t dt \\
    \end{align*}
    \begin{align*} 
        &\text{Let } u = t,\: du = dt, \: dv = \cos 3t, \: v = \frac{\sin 3t}{3}\\
        &= \sqrt{14} \Big( \Big[ \frac{ t \sin 3t}{3} \Big]_0^2 - \int_0^2 \frac{ \sin 3t}{3} dt \Big) \\
        &= \frac{\sqrt{14}}{3} \Big( 2\sin 6 + \frac{1}{3}\Big[ \cos 3t \Big]_0^2 \Big) \\
        &= \frac{\sqrt{14}}{3} \Big( 2\sin 6 + \frac{1}{3}\Big[ \cos 6 - 1 \Big] \Big) \\
        &= \frac{2\sqrt{14}}{3}\sin 6 + \frac{\sqrt{14}}{9}\cos 6 -\frac{\sqrt{14}}{9} \\
    \end{align*}
    \end{multicols}
    \item $f(x,y,z) = x + z^2 $ and $\boldsymbol{\gamma}(t) = (t,\ln t, 2\sqrt{2t})$, $1\leq t \leq 2$.
    \begin{multicols}{2} 
    \begin{align*} 
        \boldsymbol{\gamma}'(t) &= \Big( 1,\frac{1}{t},\sqrt{\frac{2}{t}} \Big) \\
        \norm{\boldsymbol{\gamma}'(t)} &= \sqrt{1 + \frac{2}{t} + \frac{1}{t^2} } \\
        \int_{\boldsymbol{\gamma}}f(x,y,z)ds &= \int_1^2 f(\boldsymbol{\gamma}(t)) \norm{\boldsymbol{\gamma}'(t)}dt \\
        &= \int_1^2 (t + 8t) \sqrt{1 + \frac{2}{t} + \frac{1}{t^2} }dt \\
        &= \int_1^2 9t \sqrt{1 + \frac{2}{t} + \frac{1}{t^2} } dt \\
        &= 9 \int_1^2 \sqrt{t^2 + 2t +  1} dt \\
    \end{align*}
    \begin{align*} 
        &= 9 \int_1^2 (t + 1) dt \\ 
        &= 9 \Big( \Big[ \frac{t^2}{2} \Big]_1^2 + \Big[1\Big]_1^2 \Big) \\
        &= 9 \Big( \frac{3}{2} + 1 \Big) \\
        &= 9 \Big( \frac{5}{2} \Big) \\
        &= \frac{45}{2} \\
    \end{align*}
    \end{multicols}
\end{enumerate}
\newpage
\item Find the average value of the following functions over the helix $\boldsymbol{\gamma}(t) = (\sin t, \cos t, t)$, $0\leq t \leq 2\pi$.
\begin{align*}
    \boldsymbol{\gamma}'(t) &= ( \cos t, - \sin t, 1) \\
    \norm{\boldsymbol{\gamma}'(t)} &= \sqrt{ \sin^2t + \cos^2t +1} \\ 
    &= \sqrt{2} \\
\end{align*}
The length of the curve $\boldsymbol{\gamma}$ is:
\begin{align*}
    \int_{\boldsymbol{\gamma}} ds &= \int_0^{2\pi} \norm{\boldsymbol{\gamma}'(t)} dt \\
    &= \sqrt{2} \int_0^{2\pi} 1 dt \\
    &= 2\sqrt{2}\pi \\
\end{align*}
So the average value of the function should just be: $\displaystyle \frac{\text{path integral}}{\text{length of }\boldsymbol{\gamma}}$
\begin{enumerate}
    \item $f(x,y,z) = y\sin z$
    \begin{align*} 
        \frac{1}{\int_0^{2\pi} \norm{\boldsymbol{\gamma}'(t)}dt}  \int_{\boldsymbol{\gamma}}f(x,y,z)ds &= \frac{1}{2\sqrt{2}\pi} \int_0^{2\pi} f(\boldsymbol{\gamma}(t)) \norm{\boldsymbol{\gamma}'(t)}dt \\
        &= \frac{1}{2\sqrt{2}\pi} \int_0^{2\pi} \cos t \sin t dt \\
        &\text{Let } u = \sin(t),\: du = \cos(t)dt \\
        &= \frac{1}{2\sqrt{2}\pi} \int_0^0 udu \\
        &= 0
    \end{align*}
    \item $f(x,y,z) = x^3 + \cos^2 z$
    \begin{align*} 
        \frac{1}{\int_0^{2\pi} \norm{\boldsymbol{\gamma}'(t)}dt}  \int_{\boldsymbol{\gamma}}f(x,y,z)ds &= \frac{1}{2\sqrt{2}\pi} \int_0^{2\pi} f(\boldsymbol{\gamma}(t)) \norm{\boldsymbol{\gamma}'(t)}dt \\
        &= \frac{1}{2\sqrt{2}\pi} \int_0^{2\pi} \sin^3t + \cos^2t dt  \\
        &= \frac{1}{2\sqrt{2}\pi} \Big( \int_0^{2\pi} (\sin t)\frac{1 - \cos^2t}{2} dt + \int_0^{2\pi} \frac{1+\cos 2t}{2} dt  \Big) \\
        &\text{Let } u = \cos(t),\: du = -\sin(t)dt \\
        &= \frac{1}{2\sqrt{2}\pi} \Big( -\int_1^{1} \frac{1 - u^2}{2} du + \int_0^{2\pi} \frac{1+\cos 2t}{2} dt  \Big) \\
        &= \frac{1}{4\sqrt{2}\pi} \Big( \int_0^{2\pi} 1+\cos 2t dt  \Big) \\
        &= \frac{1}{4\sqrt{2}\pi} \Big( 2 \pi + \Big[\frac{\sin 2t}{2} \Big]_0^{2\pi} \Big) \\
        &= \frac{1}{2\sqrt{2}} \\
    \end{align*}
\end{enumerate}
\end{enumerate}
\end{document}
