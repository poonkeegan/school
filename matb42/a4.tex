\documentclass{article}
\usepackage[utf8]{inputenc}
\usepackage[margin=1in]{geometry}
\usepackage{amsmath, amsfonts}
\usepackage{fancyhdr}
\usepackage{multicol}
\usepackage{graphicx}
\usepackage[inline]{enumitem}
\graphicspath{ {images/} }
\pagestyle{empty}
\fancyhf{}
\cfoot{\thepage}
\pagenumbering{gobble}

\lhead{MATB42: Assignment \#4 \\
Bonus}
\rhead{
Poon, Keegan\\
1002423727\\
Feb 13th 2018}
\newcommand{\norm}[1]{\| #1 \|}
\newcommand{\deriv}[1]{\frac{d}{d #1}}
\newcommand{\parti}[1]{\frac{\partial}{\partial #1}}
\renewcommand{\headrulewidth}{0pt}
\newcommand{\gam}{\boldsymbol{\gamma}}
\begin{document}

\thispagestyle{fancy}

\begin{enumerate}
    \item (For this question assume that all curves are of class $C^k$, some $k \geq 3$).

        A curve $\gam : [a,b] \rightarrow \mathbb{R}^n$ is called \textit{regular} if $\gam'(t) \not = 0$ for any t. For a regular curve $\gam$, the vector $\displaystyle \boldsymbol{T} = \frac{\gam ' (t)}{\norm{\gam '(t)}}$ is called the \textit{unit tangent vector} to the curve.
        \begin{enumerate}
            \item If $\gam : [a,b] \rightarrow \mathbb{R}^3$ is a regular curve, show that $\boldsymbol{T}'(t) \cdot \boldsymbol{T}(t) = 0$.

                (see page 235, \#16(a))

                \begin{align*}
                    \norm{\boldsymbol{T}(t)}^2 &= T_1^2 + T_2^2 + T_3^2 \\
                    \deriv{t} \norm{\boldsymbol{T}(t)}^2 &= 2 T_1T'_1 + 2T_2T'_2 + T_3T'_3 \\
                    2 (\boldsymbol{T}'(t) \cdot \boldsymbol{T}(t)) &= 2(T'_1T_1 + T'_2 T_2 + T'_3 T_3) = \deriv{t}1 = 0
                \end{align*}
        \end{enumerate}
        A curve $\gam (s)$ is said to be \textit{parameterized by arclength} (or have \textit{unit speed}) if $\norm{\gam '(s)} = 1$. The \textit{curvature} $\kappa$ at a point $\gamma(s)$ of a unit speed curve is defined by $\kappa = \norm{\boldsymbol{T}'(s)}$
        \begin{enumerate}[resume]
            \item
                \begin{enumerate}[label=(\roman*)]
                    \item If $\gam : [a,b] \rightarrow \mathbb{R}^3$ is a unit speed curve, show that its length is $b - a$.

                        The length of $\gam$ is $\displaystyle \int_{\gam} d \boldsymbol{s} = \int_a^b \norm{\gam ' (t)} \ dt$, but $\norm{\gam ' (t)}$ is 1 since $\gam$ has unit speed. Therefore, the integral is just $b-a$.

                    \item Show that $\boldsymbol{\sigma}(t) = \frac{1}{\sqrt{2}}(\cos t, \sin t, t)$ is a unit speed curve and compute its curvature.
                        
                        (see page 235, \#17)

                        \begin{align*}
                            \frac{d}{dt} \boldsymbol{\sigma}(t) &= \frac{1}{\sqrt{2}}(\deriv{t} \cos t, \deriv{t} \sin t, \deriv{t} t) \\
                            &= \frac{1}{\sqrt{2}}(- \sin t, \cos t, 1) \\
                            \implies \norm{\deriv{t} \boldsymbol{\sigma}(t)} &= \frac{\sqrt{\sin^2t+\cos^2t+1}}{\sqrt{2}} \\
                            &= \frac{\sqrt{2}}{\sqrt{2}} = 1 \quad \text{So $\boldsymbol{\sigma}(t)$ is in fact a unit curve.}
                        \end{align*}

                        Since $\boldsymbol{\sigma}(t)$ has unit speed, $\boldsymbol{T}(t)$ is just $\boldsymbol{\sigma}'(t)$, so $\boldsymbol{T}'(t)$ is $\boldsymbol{\sigma}^{(2)}(t)$.
                        \begin{align*}
                            \boldsymbol{T}'(t) &= \boldsymbol{\sigma}^{(2)}(t) \\
                            &= \frac{1}{\sqrt{2}}(\deriv{t} - \sin t,\deriv{t}  \cos t, \deriv{t} 1) \\
                            &= \frac{1}{\sqrt{2}}(- \cos t, - \sin t, 0) \\
                            \implies \norm{\boldsymbol{T}'(t)} &= \frac{1}{\sqrt{2}} = \kappa \\
                        \end{align*}
                \end{enumerate}
        \end{enumerate}
        If $\displaystyle \boldsymbol{T}'(t) \not = 0, \: \boldsymbol{N}(t) = \frac{\boldsymbol{T}'(t)}{\norm{\boldsymbol{T}'(t)}}$ is perpendicular to $\boldsymbol{T}'(t)$ (by part (a)); $\boldsymbol{N}$ is called the \textit{principal normal vector.} The vector $\boldsymbol{B}$, defined by $\boldsymbol{B} = \boldsymbol{T} \times \boldsymbol{N}$, is called the \textit{binormal vector}.
        \begin{enumerate}[resume] 
            \item Show the following about the $\boldsymbol{T}, \boldsymbol{N}$ and $\boldsymbol{B}$ system

                \begin{enumerate*}[label=(\roman*),itemjoin={\quad}]
                    \item $\displaystyle \frac{d \boldsymbol{B}}{dt} \cdot \boldsymbol{B} = 0$
                    \item $\displaystyle \frac{d \boldsymbol{B}}{dt} \cdot \boldsymbol{T} = 0$
                    \item $\displaystyle \frac{d \boldsymbol{B}}{dt}$ is a scalar multiple of $\boldsymbol{N}$.
                \end{enumerate*}

                (see page 235, \#20)

        \end{enumerate}
        If $\gam(s)$ is a unit speed curve we can define the \textit{tortion} $\tau$ by $\frac{d \boldsymbol{B}}{ds} = - \tau \boldsymbol{N}$.
        \begin{enumerate}[resume] 
            \item Compute the torsion of $\boldsymbol{\sigma}(t) = \frac{1}{\sqrt{2}}(\cos t, \sin t)$.

                (see page 235, \#21(c))
        \end{enumerate}
    \newpage
    \item Sketch the following vector fields including a few flow lines.

        \begin{enumerate*}[itemjoin={\quad}]
            \item $\boldsymbol{F}(x,y) = (1,x^2)$
            \item $\boldsymbol{F}(x,y) = (x^2,x)$
            \item $\boldsymbol{F}(x,y) = (y,-2x)$
        \end{enumerate*}
    
        \begin{enumerate}
            \item
            \begin{multicols}{3}
            \begin{align*}
                \gam(t) &= (x(t), y(t)) \\
                \gam'(t) &= (x'(t), y'(t)) \\
                \implies \frac{\frac{dy(t)}{dt}}{\frac{dx(t)}{dt}} &= \frac{y'(t)}{x'(t)} =  \frac{dy}{dx} \\
                dy &= x^2, \: dx = 3 \\
                \implies \frac{dy}{dx} &= x^2 \\
                \implies y &= \frac{x^3}{3} + c
            \end{align*}
            \end{multicols}
            \item $\boldsymbol{F}(x,y) = (x^2,x)$
            \begin{multicols}{3}
            \begin{align*}
                dy &= x, \: dx = x^2 \\
                \implies \frac{dy}{dx} &= \frac{1}{x} \\
                \implies y &= \ln{|x|} + c \quad x \not = 0
            \end{align*}
            \end{multicols}
            \item $\boldsymbol{F}(x,y) = (y,-2x)$
            \begin{multicols}{3}
            \begin{align*}
                & dy = -2x, \: dx = y \\
                \implies &\frac{dy}{dx} = \frac{-2x}{y} \\
                \implies &y\ dy = -2x\ dx \\
                \implies &\frac{y^2}{2} + x^2 = c \\
            \end{align*}
            \end{multicols}
        \end{enumerate}

    \newpage
    \item Show that the curve $\boldsymbol{c}(t) = ( t^2, 2t -1, \sqrt{t}),\ t > 0$ is a flow line of the velocity vector field $\boldsymbol{F}(x,y,z) = (y + 1,2,1/2z)$

        \[\boldsymbol{c}'(t) = \Big(2t, 2, \frac{1}{2\sqrt{t}}\Big) \]
        \[ \boldsymbol{F}(\boldsymbol{c}(t)) = \Big(2t - 1 + 1, 2, \frac{1}{2\sqrt{t}}\Big) = \Big(2t, 2, \frac{1}{2\sqrt{t}}\Big) = \boldsymbol{c}'(t) \]
        Therefore, $\boldsymbol{c}$ is a flow line of $\boldsymbol{F}$.

    \newpage
    \item Find the work done by the force field $\boldsymbol{F}(x,y,z) = (xy,yz,zx)$ in moving a particle along the twisted cubic, $\gam(t) = (t,t^2,t^3)$, from $t=0$ to $t=1$.
        \begin{align*}
            \int_{\gam} \boldsymbol{F} \cdot ds &= \int_0^1 \boldsymbol{F}(\gam(t)) \cdot \gam'(t) \ dt \\
            &= \int_0^1 (t)(t^2)(1) + (t^2)(t^3)(2t) + (t^3)(t)(3t^2)dt \\
            &= \int_0^1 t^3 + 2t^6 + 3t^6 dt\\
            &= \int_0^1 t^3 + 5t^6 \\
            &= \frac{1}{4}\Big[t^4\Big]_0^1 + \frac{5}{7}\Big[t^7\Big]_0^1\\
            &= \frac{1}{4} + \frac{5}{7} = \frac{27}{28} \\
        \end{align*}
    \newpage
    \item Evaluate each of the following integrals:
        \begin{enumerate}
                { \everymath{\displaystyle}
                    \item $\int_{\gam}xy \ dx + y^2 dy, \quad \gam(t) = (\cos t, \sin t), 0 \leq t \leq \frac{\pi}{2}$.
                        \begin{align*}
                            \int_{\gam} \omega \cdot ds &= \int_0^\frac{\pi}{2} \sin t \cos t (-\sin t) + \sin^2t \cos t \ dt \\
                            &= 0
                        \end{align*}
                    \item $\int_{\gam}\boldsymbol{F} \cdot d\boldsymbol{s}, \quad \boldsymbol{F}(x,y,z)=(y,z,x), \quad \gam(t) = \Big(t,-2t^2,\frac{1}{3}t^3\Big), \quad 0 \leq t \leq 1$.
                        \begin{align*}
                            \int_{\gam} \boldsymbol{F} \cdot ds &= \int_0^1 \boldsymbol{F}(\gam(t)) \cdot \gam'(t) \ dt \\
                            &= \int_0^1 (-2t^2)(1) + (\frac{1}{3}t^3)(-4t) + (t)(t^2)dt \\
                            &= \int_0^1 -2t^2 - \frac{4}{3}t^4 + t^3dt \\
                            &= -\frac{2}{3}\Big[t^3\Big]_0^1 - \frac{4}{15}\Big[t^5\Big]_0^1 + \frac{1}{4}\Big[t^4\Big]_0^1 \\
                            &= -\frac{10}{15}- \frac{4}{15}+ \frac{1}{4} = -\frac{56}{60} + \frac{15}{60} = -\frac{41}{60} \\
                        \end{align*}
                    \item $\int_{\gam}z \ dx -xyz\ dy + 2x^2\ dz, \quad \gam$ is the parabola $z = x^2, y=0$, from (-1,0,1) to (1,0,1).
                        
                        Can parameterize $\gam$ by $\gam(t) = (t,0,t^2), \: -1 \leq t \leq 1$, as on the parabola $y$ is constant 0, $x$ goes from $-1 \rightarrow 1$ and $z$ goes from $1 \rightarrow 0 \rightarrow 1$.
                        \begin{align*}
                            \int_{\gam} \omega \cdot ds &= \int_{-1}^{1} (t^2)(1) - (t)(0)(t^2)(0) + 2(t)^2(2t) \ dt \\
                            &= \int_{-1}^{1} t^2 + 4t^3 \ dt\\
                            &= \frac{2}{3}\Big[t^3\Big]_0^1 \quad \text{Exploiting even/odd} \\
                            &= \frac{2}{3} \\
                        \end{align*}
                    \item $\int_{\gam}\boldsymbol{F} \cdot d\boldsymbol{s}, \quad \boldsymbol{F}(x,y,z)=(2xy,x^2+e^z,ye^z), \quad \gam$ consists of straight line segments joining, in order, the points (1,1,0), (2,0,5) and (0,3,0).

                        Note: By inspection $g = x^2y + ye^z$ is a potential function for $\boldsymbol{F}$. Also, straight line segments, being linear functions are smooth. Furthermore, $F(x,y,z)$ is smooth since polynomials and exponential functions are each smooth. Therefore, GFTC applies, and $\int_{\gam} \boldsymbol{F} \cdot d \boldsymbol{s} = g(1,1,0) - g(0,3,0) = ((1)^2(1)+(1)e^{(0)}) - ((0)^2(3)+(3)e^{(0)})) = 2 - 3 = -1$.
                }
        \end{enumerate}
    \newpage 
    \item 
        \begin{enumerate}
            \item Let $\boldsymbol{F}(x,y) = (y,-x)$. Find $\displaystyle \int_{\gam} \boldsymbol{F} \cdot d\boldsymbol{s}$ from (1,0) to (0,-1) along
                \begin{enumerate}[label=(\roman*)]
                    \item the straight line segment joining these points

                        Parameterize the path as $t \mapsto (1 - t, -t)$ where $0 \leq t \leq 1$.

                        \begin{align*}
                            \int_{\gam} \boldsymbol{F} \cdot ds &= \int_0^1 \boldsymbol{F}(\gam(t)) \cdot \gam'(t) \ dt \\
                            &= \int_0^1 (-t)(-1) + (t)(-1) \ dt \\
                            &= \int_0^1 t - t \ dt = 0\\
                        \end{align*}
                    \item three-quarters of the unit circle centered at the origin traced in the counter-clockwise direction.

                        Parameterize the path as $t \mapsto (\sin -t , \cos -t) = (-\sin t, \cos t)$ where $0 \leq t \leq \frac{3\pi}{2}$. Using $-t$ since it is counter-clockwise

                        \begin{align*}
                            \int_{\gam} \boldsymbol{F} \cdot ds &= \int^{\frac{3\pi}{2}}_0 \boldsymbol{F}(\gam(t)) \cdot \gam'(t) \ dt \\
                            &= \int^{\frac{3\pi}{2}}_0 (\cos t)(-\cos t) - (-\sin t)(-\sin t) \ dt \\
                            &= \int^{\frac{3\pi}{2}}_0 -1 \ dt = -\frac{3\pi}{2}\\
                        \end{align*}

                \end{enumerate}
            \item Can your answers for part (a) help you determine if the 1-form $\omega = y\ dx - x \ dy$ is exact? Explain.

                Yes, we can determine that it is not exact. If $\omega$ were to be exact then $\boldsymbol{F}$ would be conservative implying that the line integral would be independant of path. Since the integrals are different, this is evidently not the case.

        \end{enumerate}
    \newpage
    \item Let $\boldsymbol{c}$ be the curve obtained by intersecting the cylinder $y^2 + z^2 = 4$ and the surface $x=yz$ in $\mathbb{R}^3$.
        \begin{enumerate}
            \item Give a parametrization, $\gam (t)$, of the curve $\boldsymbol{c}$.

                The cylinder simply describes a circle of radius $2$ in 2 dimensions, so $y$ and $z$ can be parameterized as $t \mapsto (2 \sin t, 2 \cos t)$. To add the additional constraint of the surface, just check what $x$ is, given the $y$ and $z$. $x = (2\sin t)(2\cos t) = 4\sin t \cos t$.
               
                Given these conditions, $\gam(t)$ is given by $(2 \sin t, 2 \cos t, 4 \sin t \cos t), \: 0 \leq t \leq 2\pi$.
            \item Evaluate $\displaystyle \int_{\boldsymbol{c}} \boldsymbol{F} \cdot d \boldsymbol{s}$, where $\boldsymbol{F}(x,y,z) = (2xy,4y,x^2)$.
                \begin{align*}
                    \int_{\gam} \boldsymbol{F} \cdot ds &= \int^{2\pi}_0 \boldsymbol{F}(\gam(t)) \cdot \gam'(t) \ dt \\
                    &= \int^{2\pi}_0 2(2 \sin t)(2 \cos t)(2 \cos t) + 4 (2 \cos t)(- 2 \sin t) + (2 \sin t)^2(4(\cos^2t - \sin^2t)) \ dt \\
                    &= \int^{2\pi}_0 16\sin t\cos^2 t - 16 \cos t \sin t  + 16 \sin^2 t \cos^2t - 16 \sin^4t \ dt \\
                    &= \int^{2\pi}_0 16(\cos^2 t - \cos t)\sin t  \ dt + \int^{2\pi}_0 16 \sin^2 t \cos^2t - 16 \sin^4t \ dt \\
                    &\text{Let $u = \cos t, \ du = -\sin t$} \\
                    &= -\int^{\cos 2\pi}_{\cos{0}} 16(u^2 - u)  \ dt + \int^{2\pi}_0 16 \sin^2 t \cos^2t - 16 \sin^4t \ dt \\
                    &= -\int^{1}_{1} 16(u^2 - u)  \ dt + \int^{2\pi}_0 16 \sin^2 t \cos^2t - 16 \sin^4t \ dt \\
                    &=  \int^{2\pi}_0 16 \sin^2 t \cos^2t - 16 \sin^4t \ dt \\
                    &=  \int^{2\pi}_0 16 \Big(\frac{(1 - \cos 2t)(1 + \cos 2t)}{4} \Big) - 16\Big(\frac{(1- \cos 2t)^2}{4} \Big) \ dt \\
                    &=  \int^{2\pi}_0 4 \Big(1 - \cos^2 2t \Big) - 4\Big(1 - 2\cos 2t+ \cos^2 2t \Big) \ dt \\
                    &=  \int^{2\pi}_0 - 8\cos^2 2t  + 8\cos 2t \ dt \\
                    &= -8 \int^{2\pi}_0 \cos^2 2t \ dt + 8 \int^{2\pi}_0 \cos 2t \ dt \\
                    &= -8 \int^{2\pi}_0 \cos^2 2t \ dt + 4 \Big[ \sin 2t\Big]_0^{2\pi} \\
                    &= -8 \int^{2\pi}_0 \cos^2 2t \ dt \\
                    &= -4 \int^{2\pi}_0 1 + \cos 4t  \ dt \\
                    &= -8\pi - \int^{2\pi}_0  \cos 4t  \ dt \\
                    &= -8\pi\\
                \end{align*}
        \end{enumerate}
\end{enumerate}
\end{document}
