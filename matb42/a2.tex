\documentclass{article}
\usepackage[utf8]{inputenc}
\usepackage[margin=1in]{geometry}
\usepackage{amsmath, amsfonts}
\usepackage{fancyhdr}
\usepackage{multicol}
\usepackage{graphicx}
\graphicspath{ {images/} }
\pagestyle{empty}
\fancyhf{}
\cfoot{\thepage}

\lhead{MATB42: Assignment \#2 \\
Bonus}
\rhead{
Poon, Keegan\\
1002423727\\
Jan 30th 2018}

\renewcommand{\headrulewidth}{0pt}
\begin{document}

\thispagestyle{fancy}

\begin{enumerate}
\item Let $\displaystyle f(x) = 
\begin{cases}
        0, &- \pi < x < -\frac{\pi}{2} \\
        2, &- \frac{\pi}{2}\leq 2 < \frac{\pi}{2} \\
        0, & \frac{\pi}{2} \leq x < \pi
\end{cases}
$
\begin{enumerate}
\item Find the Fourier series of $f$.
\begin{multicols}{2}
\noindent

\begin{align*}
a_0 &= \frac{1}{\pi} \int_{-\pi}^{\pi}f(x)dx \\
&= \frac{1}{\pi} \Bigg[ \int_{-\pi}^{-\frac{\pi}{2}}0dx + \int_{-\frac{\pi}{2}}^{\frac{\pi}{2}}2dx + \int_{\frac{\pi}{2}}^{\pi}0dx \Bigg]\\
&= \frac{1}{\pi} \Bigg[2\pi\Bigg]\\
&= 2
\end{align*}

\begin{align*}
b_k &= \frac{1}{\pi} \int_{-\pi}^{\pi}f(x)\sin(kx)dx \\
&= \frac{2}{\pi} \int_{-\frac{\pi}{2}}^{\frac{\pi}{2}}\sin(kx)dx \\
&= 0 \: \: \: [\text{sin is odd}]
\end{align*}

\begin{align*}
a_k &= \frac{1}{\pi} \int_{-\pi}^{\pi}f(x)\cos(kx)dx \\
&= \frac{2}{\pi} \int_{-\frac{\pi}{2}}^{\frac{\pi}{2}}\cos(kx)dx \\
&= \frac{2}{k\pi} \Bigg[\sin(kx)\Bigg]_{-\frac{\pi}{2}}^{\frac{\pi}{2}} \\
&= \frac{2}{k\pi} \Bigg[2\sin\Big(\frac{k\pi}{2}\Big)\Bigg]\\
&= \frac{4}{k\pi} \sin\Big(\frac{k\pi}{2}\Big)
\end{align*}
This is 0 for even elements, and alternating between 1 and -1 for odd elements.
\end{multicols}

Therefore the Fourier polynomial (for the non-zero terms) is 
\[
    1 + \sum_{l=1}^{\infty} \Bigg[ \frac{4(-1)^l}{(2l+1)\pi}\cos((2l+1)x)\Bigg]
\]
\item Determine if the Fourier series in part (a) converges. If it does converge, what are the values to which it converges (on $[-\pi, \pi]$).

The function is continuous on its partitions (they are constant functions), so by the theorem the polynomial converges to $f(x)$ on the continuous intervals. On the discontinuities, it converges to 0 at $\frac{\pi}{2}$ and $\frac{-\pi}{2}$ from the Fundemental theorem, and to 0 at $\pi$ and $-\pi$.
\item Use symbolic algeba software to sketch $f(x)$ and its $4^{th}$ degree Fourier polynomial over the interval $[-3\pi,3\pi]$.

\end{enumerate}
\newpage
\item \begin{enumerate}
\item Find the Fourier series of the function $f(x)$ defined by $\displaystyle f(x) = \begin{cases}
0 &,-\pi \leq x < 0 \\
x &, 0\leq x < \pi
\end{cases}
$ and extended from this with period $2\pi$ to all of $\mathbb{R}$.

If this Fourier series converges describe the function to which it converges.
        \begin{multicols}{2}
        \noindent
        \begin{align*}
            a_0 &= \frac{1}{\pi} \int_{-\pi}^{\pi}f(x) dx \\
            &= \frac{1}{\pi} \Bigg[\int_{-\pi}^{0}f(x) dx + \int_{0}^{\pi}f(x) dx \Bigg] \\
            &= \frac{1}{\pi} \Bigg[ 0 + \int_{0}^{\pi}x dx \Bigg] \\
            &= \frac{1}{\pi} \Bigg[\frac{1}{2}\Big[x^2\Big]_{0}^{\pi} \Bigg] \\
            &= \frac{\pi}{2}
        \end{align*}
        \begin{align*}
            a_k &= \frac{1}{\pi} \int_{-\pi}^{\pi}f(x)\cos(kx) dx \\
            &= \frac{1}{\pi} \Bigg[\int_{-\pi}^{0}0 dx + \int_{0}^{\pi}x\cos(kx) dx\Bigg] \\
            &\text{Let } u = x,\: du = dx,  \\
            & \: \: \:dv = \cos(kx),\: v = \frac{\sin(kx)}{k} \\
            &= \frac{1}{\pi} \Bigg[\frac{1}{k}\Big[x\sin(kx)\Big]^{\pi}_{0} - \frac{1}{k}\int_{0}^{\pi}\sin(kx) dx\Bigg] \\
            &= - \frac{1}{k\pi} \Bigg[\int_{0}^{\pi}\sin(kx) dx\Bigg] \\
            &= \frac{1}{k^2\pi} \Big[\cos(kx) \Big]_{0}^{\pi}\\
            &= \frac{(-1)^{-k} - 1}{k^2\pi} \\
        \end{align*}
        \begin{align*}
            b_k &= \frac{1}{\pi} \int_{-\pi}^{\pi}f(x)\sin(kx) dx \\
            &= \frac{1}{\pi} \Bigg[\int_{-\pi}^{0}0 dx +  \int_{0}^{\pi}x\sin(kx) dx \Bigg]\\
            &\text{Let } u = x,\: du = dx,\: dv = \sin(kx),\: v = -\frac{1}{k}\cos(kx) \\
            &= \frac{1}{\pi} \Bigg[-\frac{1}{k}\Big[x\cos(kx)\Big]^{\pi}_{0} + \frac{1}{k}\int_{0}^{\pi}\cos(kx) dx \Bigg]\\
            &= \frac{1}{k \pi} \Bigg[-\pi\cos(k\pi) + \frac{1}{k} \Big[ \sin(kx) \Big]_{0}^{\pi} \Bigg]\\
            &= \frac{1}{k \pi} \Bigg[-\pi\cos(k\pi) + 0 \Bigg]\\
            &= \frac{(-1)^{k+1}}{k}
        \end{align*}
        Therefore the Fourier series of $f$ is 
        \[
        F(x) = \frac{\pi}{4} + \sum^{\infty}_{k=1}\Bigg[ \frac{(-1)^k-1}{k^2\pi}\cos(kx) + \frac{(-1)^{k+1}}{k} \sin(kx) \Bigg]
        \]
        \end{multicols}
        Since $f$ is piecewise very smooth (0, $x$ are infinitely differentiable), the series converges to $f$ on $(-\pi,\pi)$ and on both endpoints, it converges to $\frac{\pi}{2}$.
        
\item Using the series from part (a) show that 
\[
\frac{\pi^2}{8} = 1 + \frac{1}{3^2} + \frac{1}{5^2} + \frac{1}{7^2} + \cdots .
\]

\begin{multicols}{2}
\noindent
\begin{align*}
        F(0) &= \frac{\pi}{4} + \sum^{\infty}_{k=1}\Bigg[ \frac{(-1)^k-1}{k^2\pi} \Bigg] \\ 
        0 &= \frac{\pi}{4} + \sum^{\infty}_{k=1}\Bigg[ \frac{-2}{(2k - 1)^2\pi} \Bigg] \\
        \end{align*}
        \begin{align*}
         \frac{\pi}{4} &= \sum^{\infty}_{k=1}\Bigg[ \frac{2}{(2k - 1)^2\pi} \Bigg] \\
         \frac{\pi^2}{8} &= \sum^{\infty}_{k=1} \frac{1}{(2k - 1)^2} \\
\end{align*}
\end{multicols}
\end{enumerate}
\newpage
\item Find the Fourier series for the restriction of the function $f(x) = 3 + 3x$ to each of the following intervals, $[a,b]$. If the Fourier series converges, to what values will the series converge at the end points?
\begin{enumerate}
\item $[a,b] = [-\pi, \pi] $
\begin{multicols}{2}
\noindent
\begin{align*}
    a_0 &= \frac{1}{\pi} \int_{-\pi}^{\pi}f(x) dx \\
    &= \frac{1}{\pi} \int_{-\pi}^{\pi}3 + 3x dx \\
    &= \frac{1}{\pi} \Bigg[6\pi +  \frac{3}{2}\Big[x^2\Big]^{\pi}_{-\pi}\Bigg] \\
    &= \frac{1}{\pi} \Bigg[6\pi + 0 \Bigg] \\
    &= 6\\
\end{align*}
\begin{align*}
    a_k &= \frac{1}{\pi} \int_{-\pi}^{\pi}f(x)\cos(kx) dx \\
    &= \frac{1}{\pi}\Bigg[ 3\int_{-\pi}^{\pi}\cos(kx) dx + \int_{-\pi}^{\pi}x\cos(kx) dx \Bigg] \\
    &= \frac{6}{\pi}\Big[\sin(kx)\Big]_{0}^{\pi} \: \: \: [\text{Since $x$ odd and cos even}] \\
    &= 0
\end{align*}
\begin{align*}
    b_k &= \frac{1}{\pi} \int_{-\pi}^{\pi}f(x)\sin(kx) dx \\
    &= \frac{3}{\pi}\Bigg[ \int_{-\pi}^{\pi}\sin(kx) dx + \int_{-\pi}^{\pi}x\sin(kx) dx \Bigg] \\
    &= \frac{6}{\pi}\Bigg[ \int_{0}^{\pi}x\sin(kx) dx \Bigg] \: \: \: [\text{Since $x$ and sin odd}] \\
    &\text{Let $u = x, du = 1 dx, dv = \sin(kx) dx, v = -\frac{\cos(kx)}{k}$} \\
    &= \frac{6}{\pi}\Bigg[ -\frac{1}{k}\Big[x\cos(kx)\Big]^{\pi}_{0} + \frac{1}{k}\int^{\pi}_{0}\cos(kx)dx \Bigg] \\
    &= \frac{6}{k\pi}\Bigg[ \pi(-1)^{k+1} +  \frac{1}{k}\Big[\sin(kx)\Big]^{\pi}_{0} \Bigg] \\
    &= \frac{6(-1)^{k+1}}{k}\\ 
    \end{align*} 
    \end{multicols}
    Linear functions are infinitely differentiable so it will converge to $f(x)$ within the interval, and coverges to 3 at the endpoints.
\item $[a,b] = [0, 2\pi]$

\begin{multicols}{2}
\noindent
\begin{align*}
    a_0 &= \frac{1}{\pi} \int_{-\pi}^{\pi}f(x) dx \\
    &= \frac{1}{\pi} \int_{0}^{2\pi}3 + 3x dx \\
    &= \frac{1}{\pi} \Bigg[6\pi +  \frac{3}{2}\Big[x^2\Big]^{2\pi}_{0}\Bigg] \\
    &= \frac{1}{\pi} \Bigg[6\pi + 6\pi^2 \Bigg] \\
    &= 6(\pi + 1)\\
\end{align*}
\begin{align*}
    a_k &= \frac{1}{\pi} \int_{-\pi}^{\pi}f(x)\cos(kx) dx \\
    &= \frac{1}{\pi}\Bigg[ 3\int_{-\pi}^{\pi}\cos(kx) dx + \int_{-\pi}^{\pi}x\cos(kx) dx \Bigg] \\
    &= \frac{6}{\pi}\Big[\sin(kx)\Big]_{0}^{\pi} \: \: \: [\text{Since $x$ odd and cos even}] \\
    &= 0
\end{align*}
\begin{align*}
    b_k &= \frac{1}{\pi} \int_{-\pi}^{\pi}f(x)\sin(kx) dx \\
    &= \frac{3}{\pi}\Bigg[ \int_{-\pi}^{\pi}\sin(kx) dx + \int_{-\pi}^{\pi}x\sin(kx) dx \Bigg] \\
    &= \frac{6}{\pi}\Bigg[ \int_{0}^{\pi}x\sin(kx) dx \Bigg] \: \: \: [\text{Since $x$ and sin odd}] \\
    &\text{Let $u = x, du = 1 dx, dv = \sin(kx) dx, v = -\frac{\cos(kx)}{k}$} \\
    &= \frac{6}{\pi}\Bigg[ -\frac{1}{k}\Big[x\cos(kx)\Big]^{\pi}_{0} + \frac{1}{k}\int^{\pi}_{0}\cos(kx)dx \Bigg] \\
    &= \frac{6}{k\pi}\Bigg[ \pi(-1)^{k+1} +  \frac{1}{k}\Big[\sin(kx)\Big]^{\pi}_{0} \Bigg] \\
    &= \frac{6(-1)^{k+1}}{k}\\ 
    \end{align*} 
    \end{multicols}
    Linear functions are infinitely differentiable so it will converge to $f(x)$ within the interval, and coverges to $3+3\pi$ at the endpoints.
\end{enumerate}

\item Find the Fourier series for the restriction of the function $f(x) = x(x-2\pi)$ and extended from this with period $2\pi$ to all of $\mathbb{R}$. Use symbolic algebra software to graph the $4^{th}$ degree Fourier polynomial together with the original function.

\item Let $f(x)$ be defined on $[0,2\pi]$ by $f(x) = x(x-2\pi)$.
\begin{enumerate}
\item Find the Fourier cosine series of $f$.
\item Find the Fourier sine series of $f$.
\item Use symbolic algebra software to graph the $4^th$ degree Fourier polynomials from parts (a) and (b) together with the original function.
\end{enumerate}
\item Find the Fourier series for the following functions:
\begin{enumerate}
\item $f(x) = \sin^2x + \sin^3x$
\item $f(x) = \sin^4x$
\item $f(x) = \cos^7x$

( \textit{Hint:} Recall that $\cos \theta = \frac{e^{i\theta} + e^{-i\theta}}{2}$ and $\sin \theta = \frac{e^{i\theta} + e^{-i\theta}}{2i}$)
\end{enumerate}

The next question is for those among you who have previously seen complex numbers. It gives another approach to Fourier series.

\item Suppose
\begin{enumerate}
\item[] {}

\begin{enumerate}
\item $f(x)$ is a real values function of $x$,
\item $\displaystyle f(x) = \sum_{n=-\infty}^\infty C_ne{inx}$ on $[-\pi, \pi ]$, where the $C_n$ are complex constants, and
\item that the term by term theorem holds true in this case
\end{enumerate}


\item Express the $C_n$ as integrals involving $f$.
\item Find the Fourier coefficients of $f$ in terms of the $C_n$.
\item Find the $C_n$ in terms of the Fourier coefficients of $f$.
\end{enumerate}
\end{enumerate}

\end{document}
