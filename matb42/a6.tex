\documentclass{article}
\usepackage[utf8]{inputenc}
\usepackage[margin=1in]{geometry}
\usepackage{amsmath, amsfonts}
\usepackage{fancyhdr}
\usepackage{multicol}
\usepackage{graphicx}
\usepackage[inline]{enumitem}
\graphicspath{ {images/} }
\pagestyle{empty}
\fancyhf{}
\cfoot{\thepage}
\pagenumbering{gobble}

\lhead{MATB42: Assignment \#6}
\rhead{
Poon, Keegan\\
1002423727\\
Mar 6th 2018}
\newcommand{\norm}[1]{\| #1 \|}
\newcommand{\deriv}[1]{\frac{d}{d #1}}
\newcommand{\parti}[1]{\frac{\partial}{\partial #1}}
\renewcommand{\headrulewidth}{0pt}
\newcommand{\gam}{\boldsymbol{\gamma}}
\begin{document}

\thispagestyle{fancy}

\begin{enumerate}
    \item Let $\displaystyle \omega = \frac{-y}{x^2+y^2}\ dx + \frac{x}{x^2+y^2}\ dy$. Calculate $\displaystyle \int_{\gam} \omega$ where
    \begin{enumerate}
        \item $\gam$ is the boundary of the triangle with vertices (in order) (0,1), (2,3) and (2,1).
        \item $\gam$ is the boundary curve of the region $\displaystyle \Bigg\{(x,y) \in \mathbb{R}^2 \Bigg|\frac{(x-2)^2}{9} + \frac{(y+1)^2}{4} \leq 1 \Bigg\}$ oriented in a counter clockwise direction
        \item $\gam$ is the graph of the polar equation $r = 3 + 2 \sin \theta$ oriented in the clockwise direction.
    \end{enumerate}
\end{enumerate}
\end{document}
