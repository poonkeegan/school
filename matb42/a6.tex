\documentclass{article}
\usepackage[utf8]{inputenc}
\usepackage[margin=1in]{geometry}
\usepackage{amsmath, amsfonts}
\usepackage{fancyhdr}
\usepackage{multicol}
\usepackage{graphicx}
\usepackage[inline]{enumitem}
\graphicspath{ {images/} }
\pagestyle{empty}
\fancyhf{}
\cfoot{\thepage}
\pagenumbering{gobble}

\lhead{MATB42: Assignment \#6}
\rhead{
Poon, Keegan\\
1002423727\\
Mar 6th 2018}
\newcommand{\norm}[1]{\| #1 \|}
\newcommand{\deriv}[1]{\frac{d}{d #1}}
\newcommand{\parti}[1]{\frac{\partial}{\partial #1}}
\renewcommand{\headrulewidth}{0pt}
\newcommand{\gam}{\boldsymbol{\gamma}}
\begin{document}

\thispagestyle{fancy}

\begin{enumerate}
    \item Let $\displaystyle \omega = \frac{-y}{x^2+y^2}\ dx + \frac{x}{x^2+y^2}\ dy$. Calculate $\displaystyle \int_{\gam} \omega$ where
    \begin{enumerate}
        \item $\gam$ is the boundary of the triangle with vertices (in order) (0,1), (2,3) and (2,1).

        The triangle does not wrap around the origin, so the winding number is 0. This means the integral should also be 0
        \item $\gam$ is the boundary curve of the region $\displaystyle \Bigg\{(x,y) \in \mathbb{R}^2 \Bigg|\frac{(x-2)^2}{9} + \frac{(y+1)^2}{4} \leq 1 \Bigg\}$ oriented in a counter clockwise direction.

        This region is a disk and 0 satifies the equation, so the winding number of its surrounding ellipse is 1 (because the boundary curve is counter clockwise). This means the integral is $1(2\pi) = 2\pi$.
        \item $\gam$ is the graph of the polar equation $r = 3 + 2 \sin \theta$ oriented in the clockwise direction.

        Again, this curve wraps around the origin once, ($r > 0$ since $-2 \leq 2 \sin \theta \leq 2$), so the integral again is $2\pi$.
    \end{enumerate}
    \newpage
    \item Let $\omega = (y^2 + z \ \ln 3)\ dx + (2xy + \sin z)\ dy + (y \cos z + (x + 1) \ln 3)\ dz$. Determine if $\omega$ is exact. If it is, use the algorithm given in class to find the potential function $g$.

    $\omega$ is exact. Given $F_1 = y^2 + z \ln 3$, $F_2 = 2xy + \sin z$, $F_3 = y \cos z + (x + 1) \ln 3$ where $\boldsymbol F = (F1,F2,F3)$
    \newpage
    \item Evaluate the double integral $\displaystyle \int_{-1}{1} \int_{-\sqrt{1-x^2}}^{\sqrt{1-x^2}}3x^2y^2dy\ dx$, by first finding an equivalent line integral.
    \newpage
    \item Let $R$ be a region in $\mathbb{R}^2$ and let $\gam$ be a counterclockwise parametrization of $\partial R$. Let $\boldsymbol F = (F_1,F-2)$ be a $C^1$ vector field defined throughout $R$ and on $\partial R$ and let $\boldsymbol n$ be the outward pointing unit normal ector to $\gam$. Use Green's theorem to give a double integral over $R$ which is equivalent to $\displaystyle \int_{\gam} \boldsymbol F \cdot \boldsymbol n \ ds$.
    \newpage
    \item Give a parametrization for each of the following surfaces, use a computer algebra sustem to plot the surface and find a unit vector normal to the surface.
    \begin{enumerate}
        \item The piece of the cylinder $y^2 + z^2 = 1$ between $x = -1$ and $x = 3$.
        \item The piece of the plane $z = x + y + 5$ which lies over the unit disk $x^2 + y^2 \leq 1$.
        \item The piece of the sphere $x^2 + y^2 + z^2 = 4$ which lies above the plane $z = 1$.
        \item The piece of the plane $x + y + z = 1$ which lies above the parallelogram: $0 \leq y - x \leq 1, 0 \leq y + x2 \leq 1$.
    \end{enumerate}
    \newpage
    \item Let $S$ be the surface given parameterically by $\boldsymbol \Phi (u,v) = (u^2, 3v, u^2 + v)$ where $(u,v) \in D$, the interior of a triangle with vertices (0,0), (3,0) and (3,3).
    \begin{enumerate}
        \item Find the surface area of $S$.
        \item Find the equation of the tangent plane to $S$ at the point (4,9,7).
    \end{enumerate}
    \newpage    
    \item Suppose the surface $S$ is the graph of a function $f : \mathbb{R}^2 \rightarrow \mathbb{R}$. Give a natural parametrization of $S$ (in terms of $f$) and derive the formula $\norm{\boldsymbol \phi_u \times \boldsymbol \phi_v} = \sqrt{1 + \norm{\text{grad }f}^2}$
    \newpage
    \item A paraboloid of revolution $S$ is parameterized by $\boldsymbol \Phi (u,v) = u \cos v, u \sin v, u^2$, $0 \leq u \leq 2, 0 \leq v \leq 2\pi$.
    \begin{enumerate}
        \item Find an equation in $x,y$ and $z$ describing the surface.
        \item What are the geometric meanings of the parameters $u$ and $v$?
        \item Find a unit vector orthogonal to the surface of $\boldsymbol \Phi (u,v)$.
        \item Find the equation for the tangent plane at $\boldsymbol \Phi(u_0,v_0) = (1,1,2)$ and express your answer in the following two ways:
        \begin{enumerate}
            \item parameterized by $u$ and $v$; and
            \item in terms of $x,y$ and $z$.
        \end{enumerate}

        \item Find the area of $S$.

        (cf. page 424, \#16)
    \end{enumerate}
    \newpage
    \item Let a differentiable function $\boldsymbol \Phi : \mathbb{R}^2 \rightarrow \mathbb{R}^3$ define a parametrized surface.
    \begin{enumerate}
        \item Assuming $\boldsymbol \phi_u \times \boldsymbol \phi_v \not = 0$, show that the range of the linear transformation $D \boldsymbol \Phi(u_0,v_0)$ is the plane spanned by $\boldsymbol \phi_u$ and $\boldsymbol \phi_v$. [Here $\boldsymbol \phi_u$ and $\boldsymbol \phi_v$ are evaluated at $(u_0,v_0)$.]
        \item Show that $\boldsymbol w \bot (\boldsymbol \phi_u \times \boldsymbol \phi_v)$ if and only if $\boldsymbol w$ is in the range of $D\boldsymbol \Phi(u_0,v_0)$.
        \item Show that the tangent plane as defined in terms of $\boldsymbol \phi_u \times \boldsymbol \phi_v (u_0,v_0)$ is the same as the "parametrized plane"
        \[
        (u,v) \mapsto \boldsymbol \Phi(u_0,v_0) + D\boldsymbol \Phi(u_0,v_0) \begin{bmatrix} u- u_0 \\ v - v_0 \end{bmatrix}
        \]

        (cf. page 383 \#20)
    \end{enumerate}
\end{enumerate}
\end{document}
